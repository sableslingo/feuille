%%% ---------------
%%% PREAMBLE
%%% ---------------
\documentclass[french,11pt]{article}

% Define geometry (without using the geometry package)
\usepackage[a4paper]{geometry}
\geometry{landscape, twocolumn, textwidth=27.5cm, textheight=19.5cm, columnsep=20mm}

%\frenchspacing						% better looking spacing

% Call packages we'll need
\usepackage{graphicx}				% images
\usepackage{multicol}
\usepackage{multirow}
\usepackage{url}					% clickable links
\usepackage{marvosym}				% symbols
\usepackage{wrapfig}				% wrapping text around figures
\usepackage{fontspec}			% font encoding
\usepackage{xunicode}
\usepackage{phonenumbers}
\usepackage[hidelinks]{hyperref}
\usepackage{ragged2e}
\usepackage{titlesec}
\usepackage{framed}
%\usepackage[default]{raleway}
\usepackage{tocvsec2}
% Customize (header and) footer
\usepackage{fancyhdr}
\usepackage{enumitem}
\usepackage{fontawesome}
\usepackage{lipsum}
\usepackage{babel}
%\usepackage{currency}
%\pagestyle{fancy}
\pagestyle{empty}
\setmainfont{Carlito}

%\newfontfamily\headingfont[]{Arial}
%\titleformat*{\section}{\Large\bfseries\sffamily}
%\titleformat*{\section}{\Large\headingfont}

%\renewcommand{\headrulewidth}{0.0pt}	% no bar on top of page
%\renewcommand{\footrulewidth}{0.4pt}	% bar on bottom of page

%%% ---------------
%%% DEFINITIONS
%%% ---------------

% Define separators

% Define Title en News input
\newcommand{\JournalName}[1]{%
		\begin{center}
			%\Huge \usefont{T1}{augie}{m}{n}
            \Large \usefont{T1}{augie}{m}{n}
			#1%
		\end{center}
		\par \normalsize \normalfont}

\newcommand*{\chants}{../chants}
\newcommand*{\messe}{../messe_900}
\newcommand*{\pu}{../pu}
\newcommand*{\psaumes}{../psaumes}
\newcommand*{\footer}{..}

%\DefineCurrency{EUR}{name={euro}, plural={euros}, symbol={\euro}, iso={EUR}, kind=iso}

\newcommand{\NewsItem}[1]{%
\vspace{3pt}
\underline{\textbf{#1}}
	%	%\usefont{T1}{augie}{m}{n}
	%	\large \textbf{#1} %\vspace{3pt}
   %     %\Large #1 \vspace{4pt}
	%	%\par
   %     \normalsize \normalfont
		  }

\newcommand{\NewsAuthor}[1]{%
			\hfill by \textsc{#1} \vspace{4pt}
			\par \normalfont}
%\sisetup{locale=FR}
%\sisetup{group-minimum-digits=3}

\graphicspath{{../images/}}

%pas de numérotation des sections
\setsecnumdepth{none}
\setlength{\parindent}{0pt}
%%% ---------------
%%% BEGIN DOCUMENT
%%% ---------------
\begin{document}

\NewsItem{CHANT D'ENTRÉE}
	\textbf{Jubilez criez de joie}

Jubilez, criez de joie. Acclamez le Dieu trois fois Saint ! Venez le prier dans la paix, témoigner de son amour.
Jubilez, criez de joie pour Dieu notre Dieu.

1 - Louez le Dieu de lumière. Il nous arrache aux ténèbres. Devenez en sa clarté des enfants de la lumière.

5 - Louange au Père et au Fils, louange à l'Esprit de gloire. Bienheureuse Trinité, notre joie et notre vie !


\NewsItem{PRÉPARATION PÉNITENTIELLE}\\
	Kyrie \emph{messe du Peuple de Dieu}


\NewsItem{GLORIA}\\
	%Messe de Lourdes
%Gloria AL 189
GLORIA ! GLORIA ! IN EXCELSIS DEO ! (bis)

Paix sur la terre aux hommes qu’il aime.
Nous te louons, nous te bénissons, nous t’adorons.
Nous te glorifions, nous te rendons grâce pour ton immense gloire,
Seigneur Dieu, roi du ciel, Dieu le Père tout-puissant.

Seigneur, Fils unique, Jésus-Christ,
Seigneur Dieu, Agneau de dieu, le fils du Père ;
Toi qui enlèves les péchés du monde, prends pitié de nous ;
Toi qui enlèves les péchés du monde, reçois notre prière ;
Toi qui es assis à la droite du Père, prends pitié de nous.

Car Toi seul est Saint, Toi seul est Seigneur,
Toi seul est le très haut ; Jésus-Christ avec le Saint Esprit
dans la gloire de Dieu le père. Amen.



% -----
\NewsItem{1\iere{} LECTURE} Si 3, 17-18.20.28-29
% -----

\NewsItem{PSAUME}
Ps 67 (68), 4-5ac, 6-7ab, 10-11

\textbf{
Béni soit le Seigneur :
il élève les humbles. 
}

\smallskip

Les justes sont en fête, ils exultent ;\\
devant la face de Dieu ils dansent de joie.\\
Chantez pour Dieu, jouez pour son nom.\\
Son nom est Le Seigneur ; dansez devant sa face.

\smallskip

Père des orphelins, défenseur des veuves,\\
tel est Dieu dans sa sainte demeure.\\
À l’isolé, Dieu accorde une maison ;\\
aux captifs, il rend la liberté.

\smallskip

Tu répandais sur ton héritage une pluie généreuse,\\
et quand il défaillait, toi, tu le soutenais.\\
Sur les lieux où campait ton troupeau,\\
tu le soutenais, Dieu qui es bon pour le pauvre.


% -----
\NewsItem{2\ieme{} LECTURE} He 12, 18-19.22-24a

\NewsItem{ACCLAMATION}
Alleluia \emph{messe de la Bienveillance}


\NewsItem{ÉVANGILE} Lc 14, 1.7-14

\NewsItem{HOMÉLIE}

\NewsItem{PROFESSION DE FOI}
%JE CROIS EN TOI PÈRE, FILS ET ESPRIT.  J’AI CONFIANCE EN TOI TU ES MON AMI

    1. Père Créateur de vie, nous sommes tes enfants tu nous donnes la vie toi qui nous aime tant.

    2. Jésus né de marie, tu es le Fils de Dieu Tu nous donnes ta vie comme un cadeau précieux.

    3. Et toi Esprit de Dieu tu nous donnes ta force un souffle silencieux nous unit, nous renforce ;

    4. Je crois que je grandis en te confiant ma vie, ma famille, mes amis au nom du Père, du Fils et de l’esprit Amen. Amen. Amen.


%\newpage

\NewsItem{PRIÈRES UNIVERSELLES}
Accueille au creux de tes mains la prière de tes enfants.


\NewsItem{OFFERTOIRE}

\NewsItem{PRIÈRES SUR LES OFFRANDES}
\textit{Nous nous levons et nous répondons : }
Que le Seigneur reçoive de vos mains ce sacrifice à la louange et à la gloire
de Son nom, pour notre bien et celui de toute l’Église.

\NewsItem{SANCTUS}
Saint le Seigneur, Saint le Seigneur,\\
Saint le Seigneur Dieu de l’univers.\\
Le ciel et la terre sont remplis de ta gloire\\
Hosanna , hosanna au plus haut des cieux !\\
Béni soit celui qui vient au nom du Seigneur\\
Hosanna, hosanna, au plus haut des cieux !


\NewsItem{ANAMNÈSE}
Nous proclamons ta mort, Seigneur ressuscité. Nous attendons ta venue dans la gloire.


\NewsItem{NOTRE PÈRE}

\NewsItem{AGNUS} \\
Agneau de Dieu Qui enlèves le péché du monde, Prends pitié de nous !  Prends pitié de nous ! (bis) \\
Agneau de Dieu Qui enlèves le péché du monde, Donne-nous la paix !  Donne-nous la paix !


\NewsItem{COMMUNION}
%Pain véritable
%D103
%Auteur : Jean Latour
%Compositeur : Robert Marthouret dit Jef
\textbf{Pain de vie, Corps ressuscité, Source vive de l’éternité.}

\begin{tabular}{p{0.5\columnwidth} p{0.5\columnwidth}}
1.
Pain véritable, Corps et Sang de Jésus Christ,\newline
Don sans réserve de l’amour du Seigneur,\newline
Corps véritable de Jésus Sauveur.
&
2.
La sainte Cêne est ici commémorée.\newline
Le même pain, le même corps sont livrés ;\newline
La sainte Cêne nous est partagée.
\end{tabular}

%3
%Pâque nouvelle désirée d’un grand désir,
%Terre promise du salut par la croix,
%Pâque éternelle éternelle joie.
%
%4
%La faim des hommes
%Dans le Christ est apaisée,
%Le pain qu’il donne est l’univers
%Consacré,
%La faim des hommes
%Pleinement comblée.
%5
%Vigne meurtrie
%Qui empourpre le pressoir,
%Que le péché ne lèse plus
%Tes rameaux,
%Vigne de gloire
%Riche en vin nouveau.
%6
%Pain de la route
%Dont le monde garde faim
%Dans la douleur et dans l’effort
%Chaque jour,
%Pain de la route
%Sois notre secours.
%7
%Vigne du Père
%Où murit un fruit divin
%Quand paraîtra le Vendangeur
%À la fin,
%Qu’auprès du Père
%Nous buvions ce Vin. 


\NewsItem{ANNONCES PAROISSIALES}


\NewsItem{CHANT D'ENVOI}\\
%PEUPLE DE DIEU, MARCHE JOYEUX
%K 180
%Didier RIMAUD
%Christian VILLENEUVE
\textbf{
Peuple de Dieu, marche joyeux, Alleluia, alleluia !
Peuple de Dieu, marche joyeux, car le Seigneur est avec toi.}

\begin{tabular}{p{0.5\columnwidth} p{0.5\columnwidth}}
1. Dieu t’a choisi parmi les peuples : pas un qu’il n’ait ainsi traité.\newline
En redisant partout son œuvre, sois le témoin de sa bonté.
&
2. Dieu t’a formé dans sa Parole et t’a fait part de son dessein :\newline
Annonce-le à tous les hommes pour qu’en son peuple ils ne soient qu’un.
\end{tabular}



\newpage

\NewsItem{Intentions de messe}
\begin{itemize}
\item[\Cross] Paulette MORNET
\end{itemize}

\NewsItem{Informations paroissiales}

\begin{tabular} {lcp{9cm}}
\multicolumn{3}{c}{\textbf{Saint Jean-Baptiste} } \\ \hline
%Mardi    & 01 juillet  & Vêpres 18h15. Messe 18h30 \\ \hline
Jeudi    & 04 sept. &
Exposition du Saint Sacrement à 16h00. Adoration. Salut au Saint Sacrement à 18h15. Messe à 18h30 
 \\ \hline
Vendredi & 05 sept. & Laudes 08h45. Messe 09h00 \\ \hline
Samedi   & 06 sept & Messe anticipée 18h00 \\ \hline
%Dimanche & & Pas de messe \\ \hline
\multicolumn{3}{c}{\textbf{Sainte Croix} } \\ \hline
%Mercredi & 02 juillet  & Messe 09h00.
%\newline \Cross{} \textbf{Enterrement}  14h30 Marie-Claire Noël \\ \hline
Dimanche  & 07 sept. & Messe 10h00\\ \hline
%\multicolumn{3}{c}{\textbf{Résidence Landsberg (3 rue Jean Monnet)} } \\ \hline
%Mercredi & 02 juillet : & Messe 10h45 \\ \hline
\end{tabular}

\begin{framed}
\begin{tabular} {lcp{7cm}}
\multicolumn{3}{c}{\textbf{Saint Jean-Baptiste} } \\
\multicolumn{3}{l}{ Du 06 juil. au 14 sept. : pas de messe le dimanche } \\
%Vendredi & 01 août & 08h45 Laudes. Pas de messe. \\
\multicolumn{3}{c}{\textbf{Sainte Croix} } \\
\multicolumn{3}{l}{ Du 06 juil. au 14 sept. : \textbf{messe unique} le dimanche à Sainte Croix à 10h00.} \\
Dimanche & 14 sept. & Messe de rentrée 10h00 \\
Dimanche & 14 sept. &  Barbecue 12h00 - 15~euros. \newline
 Inscription avant le 07 sept. auprès de Bernard Braun \texttt{bernardbrstr@gmail.com} ou \texttt{06~83~82~52~50}\\
%\multicolumn{3}{c}{\textbf{Foyer Oberlin} } \\
%\multicolumn{3}{c}{\textbf{Résidence Landsberg (3 rue Jean Monnet)} } \\
%Mercredi & 09 juillet & Messe 10h45 \\ \hline
\end{tabular}
\end{framed}

\begin{framed}
Chers parents,\\
\textbf{les inscriptions des enfants à la catéchèse} se feront en présence du Père Daniel au presbytère
les vendredis 12, 19, 26 septembre de 16h30 à 18h30. Merci de venir accompagné de votre enfant.
Plus d'informations sur internet.
\end{framed}

\NewsItem{Répétitions des chorales}
\begin{description}
\item[Chorales paroissiales] : reprise le vendredi 05 septembre (20h15 à Ste Croix)
%\item[Chorales paroissiales] : vendredi 20h15 à Sainte Croix
\end{description}

\begin{framed}
\textbf{Presbytère St Jean-Baptiste}
%2 rue de l'école 67380 Lingolsheim 03 88 78 16 45 \\
2 rue de l'école 67380 Lingolsheim \phonenumber[country=FR]{0388781645} \\
\textbf{Permanence} Lun. au Jeu. : 09h30-12h00 et 15h-18h. Ven. 10h-12h et 15h-18h. Sam. 09h30-12h00.
\textbf{Courriel} \href{mailto:danielette67380@gmail.com}{danielette67380@gmail.com}

%\textbf{Caritas} Vestiaire ouvert le mardi de 14h à 16h

\texttt{https://stjeanbaptistelingo.fr} \hfill \faFacebook Catho Lingo \hfill \faInstagram @catho\_lingo
\end{framed}



\newpage

\JournalName{Communauté de Paroisses de Lingolsheim \\
\normalsize \textit{Notre Dame des Sables}
%\\ \large \'{E}glise Saint Jean-Baptiste
\\  \normalsize \textit{22\ieme{} dimanche du Temps Ordinaire - C}
\\ \large Samedi 30 août 2025}
%\noindent\HorRule{3pt} \\[-0.75\baselineskip]
%\HorRule{1pt}
% -----

% Front article
% -----
%\vspace{0.5cm}
%	\SepRule
%\vspace{0.5cm}

%\begin{center}
\begin{minipage}[h]{1.0\linewidth}
\setlength{\parindent}{1em}
 \begin{center}
 \textbf{
 %\dots
\og 
\og Ils sont justes passés de l’autre côté\fg{}
 \fg{}
 %\dots
 }
 \end{center}

Lorsqu’un être que nous aimons, qui nous est très proche, est emporté par la mort, nous disons que nous l’avons \og perdu \fg. Et nous avons l’impression désespérante que cette perte est définitive, que la mort est vraiment une fin. Nous disons alors volontiers que nous avons \og fait une perte irréparable \fg.

En effet, c’est toujours douloureux de perdre un être cher. Nul ne peut dire le contraire. Mais chrétiens ont imbus de la parole de Dieu, la foi chrétienne nous assure que les morts ressusciteront. Nous le disons même dans notre Credo :
\og Je crois\dots{} à la résurrection de la chair, à la vie éternelle \fg{}. Cette compréhension du Credo, doit nous aider à supporter l’épreuve présente.
Nous devons le croire, même si nous ne le voyons pas. Il nous faut donc nous laisser accabler par le souvenir de la fin douloureuse de celui que nous avons perdu. La foi doit nous aider à découvrir, sous ces apparences désolantes, la croissance invisible mais réelle d’un homme nouveau, qui se dégage peu à peu de cette épreuve pour acquérir son visage de beauté et d’éternité. C’est un peu la fleur qui, alors qu’elle se dessèche et flétrit, prépare le fruit brillant et savoureux qui se noue à partir d’elle. Ainsi la mort n’est pas seulement ce qu’elle parait :
diminution et destruction. Elle est en réalité transfiguration. Et c’est bien ce que nous reprenons dans la première préface des défunts où nous disons :
\og Pour tous ceux qui croient en toi, Seigneur, la vie n’est détruite, elle est transformée ; et lorsque prend fin leur séjour sur la terre, ils ont déjà une demeure éternelle dans les cieux. \fg{}

Dans l’espérance de la mort de ceux qui nous ont devancé, si nous sommes attentifs, ne nous oblige-t-elle pas en effet à regarder les choses autrement. Ainsi pour le croyant, la vie n’est pas détruite. Elle est transformée. Notre séjour sur terre est terminé mais alors s’ouvrent pour nous les portes d’une nouvelle demeure où il n’y a plus ni larmes, ni maladie et ni souffrance ; mais la paix et la joie.

\begin{wrapfigure}{l}{1.7cm}
\vspace{-0.4cm}
	\includegraphics[scale=1.20]{../images/standing_daniel}
\end{wrapfigure}
Au cours ce mois de novembre que nous assemblées en soient déjà l’annonce et la préfiguration et que Dieu nous donne la joie de remarquer, autour de de nous, les petits signes qui montrent bien que notre prière est exaucée et que ce royaume arrive, qu’il est au milieu de nous et que bientôt nous aurons la claire vision de ce que nous espérons dans la foi.



\begin{flushright}
\textit{Père  Daniel  ETTÉ}
\end{flushright}


\end{minipage}
%\end{center}
% -----
\end{document}
