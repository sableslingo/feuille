%%% ---------------
%%% PREAMBLE
%%% ---------------
\documentclass[10pt,a4paper]{article}

% Define geometry (without using the geometry package)
\usepackage{geometry}
\geometry{landscape, twocolumn, textwidth=27.0cm, textheight=19cm, columnsep=14mm}


\frenchspacing						% better looking spacing

% Call packages we'll need
\usepackage[french]{babel}			% french
\usepackage{graphicx}				% images
\usepackage{amssymb,amsmath}		% math
\usepackage{multicol}				% three-column layout
\usepackage{url}					% clickable links
\usepackage{marvosym}				% symbols
\usepackage{wrapfig}				% wrapping text around figures
\usepackage{fontspec}			% font encoding
\usepackage{xunicode}
\usepackage{ragged2e}
\usepackage{titlesec}
\usepackage{tocvsec2}
% Customize (header and) footer
\usepackage{fancyhdr}
\usepackage{enumitem}
%\pagestyle{fancy}
\pagestyle{empty}

%\titlespacing\section{0pt}{0pt plus 4pt minus 2pt}{0pt plus 2pt minus 2pt}
%\titlespacing\subsection{0pt}{12pt plus 4pt minus 2pt}{0pt plus 2pt minus 2pt}
%\titlespacing\subsubsection{0pt}{12pt plus 4pt minus 2pt}{0pt plus 2pt minus 2pt}

%\newfontfamily\headingfont[]{Arial}
%\titleformat*{\section}{\Large\bfseries\sffamily}
%\titleformat*{\section}{\Large\headingfont}

%\renewcommand{\headrulewidth}{0.0pt}	% no bar on top of page
%\renewcommand{\footrulewidth}{0.4pt}	% bar on bottom of page

%%% ---------------
%%% DEFINITIONS
%%% ---------------

% Define separators
\newcommand{\HorRule}[1]{\noindent\rule{\linewidth}{#1}} % Creating a horizontal rule
\newcommand{\SepRule}{\noindent							 % Creating a separator
						\begin{center}
							\rule{250pt}{1pt}
						\end{center}
						}						

% Define Title en News input
\newcommand{\JournalName}[1]{%
		\begin{center}	
			%\Huge \usefont{T1}{augie}{m}{n}
            \Large \usefont{T1}{augie}{m}{n}
			#1%
		\end{center}	
		\par \normalsize \normalfont}
		
\newcommand{\JournalIssue}[1]{%
		\hfill \textsc{\mydate \today, No #1}
		\par \normalsize \normalfont}

\newcommand{\NewsItem}[1]{%
\vspace{4pt}
		%\usefont{T1}{augie}{m}{n} 	
		\large \textbf{#1} \vspace{4pt}
        %\Large #1 \vspace{4pt}
		%\par 
        \normalsize \normalfont}
		
\newcommand{\NewsAuthor}[1]{%
			\hfill by \textsc{#1} \vspace{4pt}
			\par \normalfont}		

%pas de numérotation des sections
\setsecnumdepth{none}
%%% ---------------
%%% BEGIN DOCUMENT
%%% ---------------
\begin{document}
% Title	
% -----




% Other news (1)
% -----
%\vspace{0.5cm}
%	\SepRule
%\vspace{0.5cm}

\textit{Rassemblement à la grotte pour la bénédiction des rameaux}
\NewsItem{Bénédiction des rameaux}
	Par la croix du Serviteur, porche royal où s'avancent les pécheurs\\
Par le corps de Jésus Christ, nu, outragé sous le rire des bourreaux\\
Sur les foules sans berger et sans espoir qui ne vont qu'à perdre cœur\\
Fais paraitre ton jour et le temps de ta grâce, fais paraitre ton jour, que l’homme soit sauvé

\NewsItem{Évangile} Luc 19,28-40
\NewsItem{Entrée en procession}

	\NewsItem{Chant d'entrée}
	%\section{Chant d'entrée}
	\begin{itemize}
\item[R/] Hosanna, Hosanna, Hosanna au plus haut des cieux !
\item[]
Saint, saint, saint, le Seigneur, Dieu de l'univers.
\item[]
Le ciel et la terre sont remplis de ta gloire. R/
\item[]
Béni soit celui qui vient au nom du Seigneur. R/
\end{itemize}

\NewsItem{Chant d'entrée}

% -----
%\section{Préparation pénitentielle}
\NewsItem{Préparation pénitentielle}
Kyrie \emph{messe du Peuple de Dieu}

% -----

\NewsItem{Gloire à Dieu}  
%Messe de Lourdes
%Gloria AL 189
GLORIA ! GLORIA ! IN EXCELSIS DEO ! (bis)

Paix sur la terre aux hommes qu’il aime.
Nous te louons, nous te bénissons, nous t’adorons.
Nous te glorifions, nous te rendons grâce pour ton immense gloire,
Seigneur Dieu, roi du ciel, Dieu le Père tout-puissant.

Seigneur, Fils unique, Jésus-Christ,
Seigneur Dieu, Agneau de dieu, le fils du Père ;
Toi qui enlèves les péchés du monde, prends pitié de nous ;
Toi qui enlèves les péchés du monde, reçois notre prière ;
Toi qui es assis à la droite du Père, prends pitié de nous.

Car Toi seul est Saint, Toi seul est Seigneur,
Toi seul est le très haut ; Jésus-Christ avec le Saint Esprit
dans la gloire de Dieu le père. Amen.


% -----
\NewsItem{Première lecture}
\og Me voici : envoie-moi ! \fg (Is 6, 1-2a.3-8)
% -----

\NewsItem{Psaume}
Ps 137 (138), 1-2a, 2bc-3, 4-5, 7c-8
\begin{itemize}
\item[R/] Je te chante, Seigneur, en présence des anges.
\item
De tout mon cœur, Seigneur, je te rends grâce :
tu as entendu les paroles de ma bouche.
Je te chante en présence des anges,
vers ton temple sacré, je me prosterne.
\item
Je rends grâce à ton nom pour ton amour et ta vérité,
car tu élèves, au-dessus de tout, ton nom et ta parole.
Le jour où tu répondis à mon appel,
tu fis grandir en mon âme la force.
\item
Tous les rois de la terre te rendent grâce
quand ils entendent les paroles de ta bouche.
Ils chantent les chemins du Seigneur :
\og Qu’elle est grande, la gloire du Seigneur ! \fg
\item
Ta droite me rend vainqueur.
Le Seigneur fait tout pour moi !
Seigneur, éternel est ton amour :
n’arrête pas l’œuvre de tes mains.
\end{itemize}



% -----
\NewsItem{Deuxième lecture}
\og Voilà ce que nous proclamons, voilà ce que vous croyez \fg (1 Co 15, 1-11)

% -----

\NewsItem{Acclamation} : Alléluia, Alléluia ! (4 fois)
% -----

% -----
\NewsItem{Évangile} : \og Laissant tout, ils le suivirent \fg (Lc 5, 1-11)
% -----


\NewsItem{Profession de foi} 


\NewsItem{Prières universelles} 
Entends Seigneur la prière, qui monte de nos cœurs.
\newpage

\NewsItem{Offertoire} 
Venez à moi, vous qui portez un fardeau 
\begin{itemize}
\item[R/] Venez à moi, vous qui portez un fardeau. Venez, vous tous qui peinez. 
     Et moi, je vous soulagerai. Je suis le repos de vos âmes.
\item[1.]  Mettez-vous à mon école, car je suis doux, je suis humble de cœur. Prenez 
      mon joug, il est aisé et vous trouverez la paix. Mon fardeau est léger !
\item[2.]
Devant toi je tiens mon âme, comme un enfant dans les bras de sa mère. Seigneur, mon âme espère en toi ! En silence et dans la foi, j'espère le Seigneur !
\end{itemize}

\NewsItem{Prières sur les offrandes}
\textit{Nous nous levons et nous répondons : }

Que le Seigneur reçoive de vos mains ce sacrifice à la louange et à la gloire 
de Son nom, pour notre bien et celui de toute l’Église.

\NewsItem{Sanctus}
\begin{itemize}
\item[R/] Trois fois Saint, trois fois Saint, le Seigneur Dieu de l’univers. 
      Hosanna, hosanna (bis) au plus haut des Cieux ! 
\item[1.]  Le Ciel et la Terre nous chantent Ta gloire, hosanna au plus haut des Cieux. 
      Béni soit Celui qui vient, c’est Jésus notre Sauveur ! 
\end{itemize}

\NewsItem{Anamnèse}
Gloire à Toi qui étais mort. Gloire à Toi qui est vivant.  
Notre Sauveur et notre Dieu, viens, Seigneur Jésus.

\NewsItem{Notre Père}

\NewsItem{Geste de paix}
Donne la paix, Seigneur, donne Ta paix ! (bis) 

\NewsItem{Agnus}
\begin{itemize}
    \item 
    Agneau de Dieu qui enlèves les péchés du monde, prends pitié de nous (bis)
    \item
Agneau de Dieu qui enlèves les péchés du monde, donne-nous la paix
\end{itemize}

\NewsItem{Communion} Pour former un seul corps
\begin{itemize}
    \item [1.]
    Pour former un seul corps, boire à la même coupe, pour former un
       seul corps, comme des milliers de grains ne font qu’un bout de pain. 
       Pour former un seul corps, boire à la même coupe, pour former un seul 
       corps et que l’on soit d’accord pour que règne l’amour…
\item[R/]  Mangeons ce pain, le pain vivant, buvons ce vin qui est son sang, corps               
       et sang de Jésus-Christ, pour suivre son chemin et devenir témoins.

\item[2.]    Pour former un seul corps, boire à la même coupe, pour former un 
        seul corps, comme des milliers de grains n’offrent qu’un peu de vin. 
        Pour former un seul corps, boire à la même coupe, pour former un seul 
        corps, donner chacun de soi pour que règne la joie…
\end{itemize}

\newpage

\NewsItem{Envoi} Allons dire partout comment Jésus est bon
\begin{itemize}
    \item [R/] Allons dire partout comment Jésus est bon (bis) oui chantons partout, ne gardons 
     pas ça pour nous, allons chanter partout les merveilles qu’il fait pour nous !
\item[1.]  Il me fait découvrir comme la vie en Lui est jolie, quand je marche avec Lui, quand je marche 
     avec Lui. Il me fait retrouver la confiance que j’avais perdue, me rend tellement content,  
     que je chante son nom tout le temps !
\item[3.] Il me fait découvrir comment l’amour en Lui fleurit, quand je suis son chemin, quand je suis 
     son chemin. Il me fait retrouver toute la joie que j’avais perdue, me rend tellement content,  
     que je chante son nom tout le temps !
     \end{itemize}

     \NewsItem{Informations paroissiales}
             
         \begin{tabular}{l l l}
         \multicolumn{3}{c}{\textbf{St Jean-Baptiste}} \\
  Mardi & 11 fév. & Vêpres 18h15 - 18h30. Pas de messe \\
Jeudi & 13 fév. & Salut au Saint Sacrement 18h15. Messe 18h30. \\
    Vendredi & 14 fév. & Laudes 08h45 - 09h00. Pas de messe \\
        Samedi  & 15 fév. & Messe anticipée 18h00 \\
    Dimanche & 16 fév. & Pas de messe \\      
      
         \multicolumn{3}{c}{\textbf{Ste Croix}} \\
         Mercredi & 12 fév. & Pas de messe \\ 
         Dimanche & 16 fév.& Messe 10h30 \\
    
        \end{tabular}
  

\newpage

\JournalName{Communauté de Paroisses de Lingolsheim \\
\normalsize \textit{Notre Dame des Sables}
\\ \large \'{E}glise Saint Jean-Baptiste
\\  \normalsize \textit{5ème dimanche du Temps Ordinaire - année C}
\\ \large Samedi 08 février 2025 à 18h00}
%\noindent\HorRule{3pt} \\[-0.75\baselineskip]
%\HorRule{1pt}
% -----

% Front article
% -----
%\vspace{0.5cm}
%	\SepRule
%\vspace{0.5cm}

%\begin{center}
\begin{minipage}[h]{1.0\linewidth}
 \begin{center}
 \textbf{
 %\dots
\og 
\og Ils sont justes passés de l’autre côté\fg{}
 \fg{}
 %\dots
 }
 \end{center}

Lorsqu’un être que nous aimons, qui nous est très proche, est emporté par la mort, nous disons que nous l’avons \og perdu \fg. Et nous avons l’impression désespérante que cette perte est définitive, que la mort est vraiment une fin. Nous disons alors volontiers que nous avons \og fait une perte irréparable \fg.

En effet, c’est toujours douloureux de perdre un être cher. Nul ne peut dire le contraire. Mais chrétiens ont imbus de la parole de Dieu, la foi chrétienne nous assure que les morts ressusciteront. Nous le disons même dans notre Credo :
\og Je crois\dots{} à la résurrection de la chair, à la vie éternelle \fg{}. Cette compréhension du Credo, doit nous aider à supporter l’épreuve présente.
Nous devons le croire, même si nous ne le voyons pas. Il nous faut donc nous laisser accabler par le souvenir de la fin douloureuse de celui que nous avons perdu. La foi doit nous aider à découvrir, sous ces apparences désolantes, la croissance invisible mais réelle d’un homme nouveau, qui se dégage peu à peu de cette épreuve pour acquérir son visage de beauté et d’éternité. C’est un peu la fleur qui, alors qu’elle se dessèche et flétrit, prépare le fruit brillant et savoureux qui se noue à partir d’elle. Ainsi la mort n’est pas seulement ce qu’elle parait :
diminution et destruction. Elle est en réalité transfiguration. Et c’est bien ce que nous reprenons dans la première préface des défunts où nous disons :
\og Pour tous ceux qui croient en toi, Seigneur, la vie n’est détruite, elle est transformée ; et lorsque prend fin leur séjour sur la terre, ils ont déjà une demeure éternelle dans les cieux. \fg{}

Dans l’espérance de la mort de ceux qui nous ont devancé, si nous sommes attentifs, ne nous oblige-t-elle pas en effet à regarder les choses autrement. Ainsi pour le croyant, la vie n’est pas détruite. Elle est transformée. Notre séjour sur terre est terminé mais alors s’ouvrent pour nous les portes d’une nouvelle demeure où il n’y a plus ni larmes, ni maladie et ni souffrance ; mais la paix et la joie.

\begin{wrapfigure}{l}{1.7cm}
\vspace{-0.4cm}
	\includegraphics[scale=1.20]{../images/standing_daniel}
\end{wrapfigure}
Au cours ce mois de novembre que nous assemblées en soient déjà l’annonce et la préfiguration et que Dieu nous donne la joie de remarquer, autour de de nous, les petits signes qui montrent bien que notre prière est exaucée et que ce royaume arrive, qu’il est au milieu de nous et que bientôt nous aurons la claire vision de ce que nous espérons dans la foi.



\begin{flushright}
\textit{Père  Daniel  ETTÉ}
\end{flushright}


\end{minipage}
%\end{center}
% -----


% Other news (2)
% -----
%\section{Elephant eats frog}
%\NewsAuthor{J. Doe}
%	\blindtext[1]
%		\begin{center}
%			\includegraphics[width=0.8\linewidth]{elephant}
%		\end{center}
%		\blindtext[1]

% -----
\end{document} 
