%Pain véritable
%D103
%Auteur : Jean Latour
%Compositeur : Robert Marthouret dit Jef
\textbf{Pain de vie, Corps ressuscité, Source vive de l’éternité.}

\begin{tabular}{p{0.5\columnwidth} p{0.5\columnwidth}}
1.
Pain véritable, Corps et Sang de Jésus Christ,\newline
Don sans réserve de l’amour du Seigneur,\newline
Corps véritable de Jésus Sauveur.
&
2.
La sainte Cêne est ici commémorée.\newline
Le même pain, le même corps sont livrés ;\newline
La sainte Cêne nous est partagée.
\end{tabular}

%3
%Pâque nouvelle désirée d’un grand désir,
%Terre promise du salut par la croix,
%Pâque éternelle éternelle joie.
%
%4
%La faim des hommes
%Dans le Christ est apaisée,
%Le pain qu’il donne est l’univers
%Consacré,
%La faim des hommes
%Pleinement comblée.
%5
%Vigne meurtrie
%Qui empourpre le pressoir,
%Que le péché ne lèse plus
%Tes rameaux,
%Vigne de gloire
%Riche en vin nouveau.
%6
%Pain de la route
%Dont le monde garde faim
%Dans la douleur et dans l’effort
%Chaque jour,
%Pain de la route
%Sois notre secours.
%7
%Vigne du Père
%Où murit un fruit divin
%Quand paraîtra le Vendangeur
%À la fin,
%Qu’auprès du Père
%Nous buvions ce Vin. 
