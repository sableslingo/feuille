%%% ---------------
%%% PREAMBLE
%%% ---------------
\documentclass[11pt,a4paper]{article}

% Define geometry (without using the geometry package)
\usepackage{geometry}
\geometry{landscape, twocolumn, textwidth=27.5cm, textheight=19.5cm, columnsep=15mm}

%\frenchspacing						% better looking spacing

% Call packages we'll need
\usepackage[french]{babel}			% french
\usepackage{graphicx}				% images
\usepackage{amssymb,amsmath}		% math
\usepackage{multicol}				% three-column layout
\usepackage{url}					% clickable links
\usepackage{marvosym}				% symbols
\usepackage{wrapfig}				% wrapping text around figures
\usepackage{fontspec}			% font encoding
\usepackage{xunicode}
\usepackage{ragged2e}
\usepackage{titlesec}
\usepackage{framed}
\usepackage{tocvsec2}
% Customize (header and) footer
\usepackage{fancyhdr}
\usepackage{enumitem}
%\pagestyle{fancy}
\pagestyle{empty}
\setmainfont{Carlito}

%\titlespacing\section{0pt}{0pt plus 4pt minus 2pt}{0pt plus 2pt minus 2pt}
%\titlespacing\subsection{0pt}{12pt plus 4pt minus 2pt}{0pt plus 2pt minus 2pt}
%\titlespacing\subsubsection{0pt}{12pt plus 4pt minus 2pt}{0pt plus 2pt minus 2pt}

%\newfontfamily\headingfont[]{Arial}
%\titleformat*{\section}{\Large\bfseries\sffamily}
%\titleformat*{\section}{\Large\headingfont}

%\renewcommand{\headrulewidth}{0.0pt}	% no bar on top of page
%\renewcommand{\footrulewidth}{0.4pt}	% bar on bottom of page

%%% ---------------
%%% DEFINITIONS
%%% ---------------

% Define separators
\newcommand{\HorRule}[1]{\noindent\rule{\linewidth}{#1}} % Creating a horizontal rule
\newcommand{\SepRule}{\noindent							 % Creating a separator
						\begin{center}
							\rule{250pt}{1pt}
						\end{center}
						}						

% Define Title en News input
\newcommand{\JournalName}[1]{%
		\begin{center}	
			%\Huge \usefont{T1}{augie}{m}{n}
            \Large \usefont{T1}{augie}{m}{n}
			#1%
		\end{center}	
		\par \normalsize \normalfont}
		
\newcommand{\JournalIssue}[1]{%
		\hfill \textsc{\mydate \today, No #1}
		\par \normalsize \normalfont}

\newcommand{\NewsItem}[1]{%
\vspace{3pt}
\underline{\textbf{#1}}
	%	%\usefont{T1}{augie}{m}{n} 	
	%	\large \textbf{#1} %\vspace{3pt}
   %     %\Large #1 \vspace{4pt}
	%	%\par 
   %     \normalsize \normalfont
		  }
		
\newcommand{\NewsAuthor}[1]{%
			\hfill by \textsc{#1} \vspace{4pt}
			\par \normalfont}		

%pas de numérotation des sections
\setsecnumdepth{none}
\setlength{\parindent}{0pt}
%%% ---------------
%%% BEGIN DOCUMENT
%%% ---------------
\begin{document}
% Title	
% -----




% Other news (1)
% -----
%\vspace{0.5cm}
%	\SepRule
%\vspace{0.5cm}

\textit{Rassemblement à la grotte pour la bénédiction des rameaux}

\NewsItem{BÉNÉDICTION DES RAMEAUX}

	Par la croix du Serviteur, porche royal où s'avancent les pécheurs\\
Par le corps de Jésus Christ, nu, outragé sous le rire des bourreaux\\
Sur les foules sans berger et sans espoir qui ne vont qu'à perdre cœur\\
Fais paraitre ton jour et le temps de ta grâce, fais paraitre ton jour, que l’homme soit sauvé


\NewsItem{ÉVANGILE} Lc 19, 28-40

\NewsItem{ENTRÉE EN PROCESSION}

\NewsItem{CHANT D'ENTRÉE}
	%\section{Chant d'entrée}
	\begin{itemize}
\item[R/] Hosanna, Hosanna, Hosanna au plus haut des cieux !
\item[]
Saint, saint, saint, le Seigneur, Dieu de l'univers.
\item[]
Le ciel et la terre sont remplis de ta gloire. R/
\item[]
Béni soit celui qui vient au nom du Seigneur. R/
\end{itemize}



% -----
\NewsItem{PREMIÈRE LECTURE} Is 50, 4-7
% -----

\NewsItem{PSAUME} (21 (22), 8-9, 17-18a, 19-20, 22c-24a)
\begin{itemize}
\item[R/]
Mon Dieu, mon Dieu,
pourquoi m’as-tu abandonné ? (Ps 21, 2a)
\item[]
Tous ceux qui me voient me bafouent ;
ils ricanent et hochent la tête :
\og Il comptait sur le Seigneur : qu’il le délivre !
Qu’il le sauve, puisqu’il est son ami ! \fg
\item[]
Oui, des chiens me cernent,
une bande de vauriens m’entoure ;
Ils me percent les mains et les pieds,
je peux compter tous mes os. R/
\item[]
Ils partagent entre eux mes habits
et tirent au sort mon vêtement.
Mais toi, Seigneur, ne sois pas loin :
ô ma force, viens vite à mon aide !
\item[]
Tu m’as répondu !
Et je proclame ton nom devant mes frères,
je te loue en pleine assemblée.
Vous qui le craignez, louez le Seigneur. R/
\end{itemize}


% -----
\NewsItem{DEUXIÈME LECTURE} Ph 2 6-11

\NewsItem{ACCLAMATION}
	%\begin{itemize}
%\item[]
%Gloire au Christ parole éternelle du Dieu vivant !
%\item[]
%Gloire à toi Seigneur !
%\end{itemize}
Gloire au Christ parole éternelle du Dieu vivant !
Gloire à toi Seigneur !


% -----

\NewsItem{ÉVANGILE} Passion de notre Seigneur Jésus Christ (Lc 22, 14 – 23, 56)

\texttt{Chant durant la passion}  R. : Mon peuple, que t'ai-je fait ? En quoi
t'ai-je offensé ? Réponds-moi ! Ô Dieu Saint, ô Dieu Saint, fort.
Ô Dieu Saint, ô Dieu Saint, fort, immortel, prends pitié de nous.


\NewsItem{HOMÉLIE}

\NewsItem{PROFESSION DE FOI} 

%\newpage

\NewsItem{PRIÈRES UNIVERSELLES} 
O Christ mort sur la croix, exauce notre prière

\NewsItem{OFFERTOIRE} 
%Venez à moi, vous qui portez un fardeau 
%\begin{itemize}
%\item[R/] Venez à moi, vous qui portez un fardeau. Venez, vous tous qui peinez. 
%     Et moi, je vous soulagerai. Je suis le repos de vos âmes.
%\item[1.]  Mettez-vous à mon école, car je suis doux, je suis humble de cœur. Prenez 
%      mon joug, il est aisé et vous trouverez la paix. Mon fardeau est léger !
%\item[2.]
%Devant toi je tiens mon âme, comme un enfant dans les bras de sa mère. Seigneur, mon âme espère en toi ! En silence et dans la foi, j'espère le Seigneur !
%\end{itemize}

\NewsItem{PRIÈRES SUR LES OFFRANDES}
\textit{Nous nous levons et nous répondons : }
Que le Seigneur reçoive de vos mains ce sacrifice à la louange et à la gloire 
de Son nom, pour notre bien et celui de toute l’Église.

\NewsItem{SANCTUS}
Saint, saint, saint, le Seigneur, Dieu de l’univers.
Le ciel et la terre sont remplis de ta gloire. Hosanna au plus haut des cieux !
Béni soit celui qui vient au nom du Seigneur. Hosanna au plus haut des cieux !

%\begin{itemize}
%\item[R/] Trois fois Saint, trois fois Saint, le Seigneur Dieu de l’univers. 
%      Hosanna, hosanna (bis) au plus haut des Cieux ! 
%\item[1.]  Le Ciel et la Terre nous chantent Ta gloire, hosanna au plus haut des Cieux. 
%      Béni soit Celui qui vient, c’est Jésus notre Sauveur ! 
%\end{itemize}

\NewsItem{ANAMNÈSE}
Gloire à Toi qui étais mort. Gloire à Toi qui est vivant.  
Notre Sauveur et notre Dieu, viens, Seigneur Jésus.

\NewsItem{NOTRE PÈRE}

\NewsItem{AGNUS}

Agneau de Dieu qui enlèves les péchés du monde, prends pitié de nous\\
Agneau de Dieu qui enlèves les péchés du monde, prends pitié de nous\\
Agneau de Dieu qui enlèves les péchés du monde, donne-nous la paix

\NewsItem{COMMUNION}
Montre-nous ton visage d’amour
\begin{itemize}
\item[R/] Montre-nous ton visage d’amour, Dieu très bon, à jamais fidèle.
Conduis-nous jusque dans ta maison, donne-nous ta joie éternelle.
\item[1.]
En toi est né l'univers, fruit de ta beauté. Nous bénissons ta splendeur,
par Jésus Premier-né, dans l'amour de l'Esprit Saint. R/
\item[2.]
Nous sommes nés en ton cœur, Amour éternel. Vers toi revient notre vie,
par Jésus Premier-né, dans l'amour de l'Esprit Saint. R/
\item[3.]
Source de joie et de paix, ô soleil d'amour, tu es tendresse infinie,
par Jésus Premier-né, dans l'amour de l'Esprit Saint. R/
\item[4.]
Quand tu viendras nous juger au dernier matin, prends tes enfants dans
tes bras, par Jésus Premier-né, dans l'amour de l'Esprit Saint. R/
\end{itemize}


\NewsItem{CHANT D'ENVOI}
Au cœur de nos détresses
\begin{enumerate}
\item
Au cœur de nos détresses, aux cris de nos douleurs
C’est toi qui souffres sur nos croix et nous passons sans te voir (bis).
\item
Au vent de nos tempêtes, au souffle des grands froids
C’est toi qui doutes sur nos croix et nous passons sans te voir (bis).
\end{enumerate}


\newpage


\NewsItem{Informations paroissiales}

Les horaires des messes et des célébrations sont affichés à l’extérieur des églises, 
sur Internet et les réseaux sociaux Facebook et Instagram.          



\begin{framed}
\begin{description}
\item[Messe Chrismale]
Mardi 15 avril à 18h30 à la cathédrale de Strasbourg
\item[Célébrations de Pâques]
~\\
Jeudi Saint 17 avril : Messe unique à 20h à l’église Sainte-Croix\\
Vendredi Saint 18 avril :
\begin{itemize}
\item[]
Chemin de Croix à 13h30 dans l’église Saint Jean-Baptiste
\item[]
Célébration de la Passion à 15h à l’église Saint Jean-Baptiste
\end{itemize}
Samedi 19 avril              : Vigile Pascale à 20h30 à l’église Sainte-Croix\\
Dimanche 20 avril         : Messe de Pâques à 10h à l’église Saint Jean-Baptiste
\item[Rencontre des confirmands avec Mgr Pascal DELANNOY]
Mardi 22 avril de 18h30 à 20h à l’Archevêché 16 Rue Brûlée, 67000 Strasbourg
\item[Voyage en Côte d’Ivoire du 8 au 17 février 2026]
~\\
Mercredi 23 avril : réunion d’information concernant le voyage en Côte d’Ivoire
organisé avec le Diocèse de Grand Bassam à un tarif attractif.
Inscription par mail avant \underline{le 30 avril 2025} à : \texttt{paule.troestler@gmail.com} et/ou   \texttt{nathalie\_dick@yahoo.fr}
\item[Autre]
\end{description}
\end{framed}


\textbf{Répétitions des chorales}
\begin{description}
\item[Chorales paroissiales] : vendredi 20h15 à Sainte Croix
\item[Groupe « Dans l’Amour de Dieu »] : samedi 16h30 à Saint Jean-Baptiste
\end{description}

\begin{framed}
\begin{description}
\item[Presbytère Saint Jean-Baptiste]
~\\
2 rue de l'école 67380 Lingolsheim 03 88 78 16 45
\item[Permanence] Vendredi de 10h à 12h et de 17h30 à 19h
\item[Courriel] \texttt{danielette67380@gmail.com}
\item[site internet] \texttt{stjeanbaptistelingo.fr}
\item[Caritas] Vestiaire ouvert le mardi de 14h à 16h
\end{description}
\end{framed}

%         \begin{tabular}{l l l}
%         \multicolumn{3}{c}{\textbf{St Jean-Baptiste}} \\
%  Mardi & 11 fév. & Vêpres 18h15 - 18h30. Pas de messe \\
%Jeudi & 13 fév. & Salut au Saint Sacrement 18h15. Messe 18h30. \\
%    Vendredi & 14 fév. & Laudes 08h45 - 09h00. Pas de messe \\
%        Samedi  & 15 fév. & Messe anticipée 18h00 \\
%    Dimanche & 16 fév. & Pas de messe \\      
%      
%         \multicolumn{3}{c}{\textbf{Ste Croix}} \\
%         Mercredi & 12 fév. & Pas de messe \\ 
%         Dimanche & 16 fév.& Messe 10h30 \\
%    
%        \end{tabular}
  

\newpage

\JournalName{Communauté de Paroisses de Lingolsheim \\
\normalsize \textit{Notre Dame des Sables}
\\ \large \'{E}glise Saint Jean-Baptiste
\\  \normalsize \textit{Dimanche des Rameaux et de la Passion du Seigneur}
\\ \large Dimanche 13 avril  2025}
%\noindent\HorRule{3pt} \\[-0.75\baselineskip]
%\HorRule{1pt}
% -----

% Front article
% -----
%\vspace{0.5cm}
%	\SepRule
%\vspace{0.5cm}

%\begin{center}
\begin{minipage}[h]{1.0\linewidth}
 \begin{center}
 \textbf{
 %\dots
\og 
\og Ils sont justes passés de l’autre côté\fg{}
 \fg{}
 %\dots
 }
 \end{center}

Lorsqu’un être que nous aimons, qui nous est très proche, est emporté par la mort, nous disons que nous l’avons \og perdu \fg. Et nous avons l’impression désespérante que cette perte est définitive, que la mort est vraiment une fin. Nous disons alors volontiers que nous avons \og fait une perte irréparable \fg.

En effet, c’est toujours douloureux de perdre un être cher. Nul ne peut dire le contraire. Mais chrétiens ont imbus de la parole de Dieu, la foi chrétienne nous assure que les morts ressusciteront. Nous le disons même dans notre Credo :
\og Je crois\dots{} à la résurrection de la chair, à la vie éternelle \fg{}. Cette compréhension du Credo, doit nous aider à supporter l’épreuve présente.
Nous devons le croire, même si nous ne le voyons pas. Il nous faut donc nous laisser accabler par le souvenir de la fin douloureuse de celui que nous avons perdu. La foi doit nous aider à découvrir, sous ces apparences désolantes, la croissance invisible mais réelle d’un homme nouveau, qui se dégage peu à peu de cette épreuve pour acquérir son visage de beauté et d’éternité. C’est un peu la fleur qui, alors qu’elle se dessèche et flétrit, prépare le fruit brillant et savoureux qui se noue à partir d’elle. Ainsi la mort n’est pas seulement ce qu’elle parait :
diminution et destruction. Elle est en réalité transfiguration. Et c’est bien ce que nous reprenons dans la première préface des défunts où nous disons :
\og Pour tous ceux qui croient en toi, Seigneur, la vie n’est détruite, elle est transformée ; et lorsque prend fin leur séjour sur la terre, ils ont déjà une demeure éternelle dans les cieux. \fg{}

Dans l’espérance de la mort de ceux qui nous ont devancé, si nous sommes attentifs, ne nous oblige-t-elle pas en effet à regarder les choses autrement. Ainsi pour le croyant, la vie n’est pas détruite. Elle est transformée. Notre séjour sur terre est terminé mais alors s’ouvrent pour nous les portes d’une nouvelle demeure où il n’y a plus ni larmes, ni maladie et ni souffrance ; mais la paix et la joie.

\begin{wrapfigure}{l}{1.7cm}
\vspace{-0.4cm}
	\includegraphics[scale=1.20]{../images/standing_daniel}
\end{wrapfigure}
Au cours ce mois de novembre que nous assemblées en soient déjà l’annonce et la préfiguration et que Dieu nous donne la joie de remarquer, autour de de nous, les petits signes qui montrent bien que notre prière est exaucée et que ce royaume arrive, qu’il est au milieu de nous et que bientôt nous aurons la claire vision de ce que nous espérons dans la foi.



\begin{flushright}
\textit{Père  Daniel  ETTÉ}
\end{flushright}


\end{minipage}
%\end{center}
% -----
\end{document} 
