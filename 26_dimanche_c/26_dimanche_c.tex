%%% ---------------
%%% PREAMBLE
%%% ---------------
\documentclass[french,11pt]{article}

% Define geometry (without using the geometry package)
\usepackage[a4paper]{geometry}
\geometry{landscape, twocolumn, textwidth=28.0cm, textheight=19.5cm, columnsep=17mm}

%\frenchspacing						% better looking spacing

% Call packages we'll need
\usepackage{graphicx}				% images
\usepackage{multicol}
\usepackage{multirow}
\usepackage{url}					% clickable links
\usepackage{marvosym}				% symbols
\usepackage{wrapfig}				% wrapping text around figures
\usepackage{fontspec}			% font encoding
\usepackage{xunicode}
\usepackage{phonenumbers}
\usepackage[hidelinks]{hyperref}
\usepackage{ragged2e}
\usepackage{titlesec}
\usepackage{framed}
%\usepackage[default]{raleway}
\usepackage{tocvsec2}
% Customize (header and) footer
\usepackage{fancyhdr}
\usepackage{enumitem}
\usepackage{fontawesome}
\usepackage{lipsum}
\usepackage{babel}
%\usepackage{currency}
%\pagestyle{fancy}
\pagestyle{empty}
\setmainfont{Carlito}

%\newfontfamily\headingfont[]{Arial}
%\titleformat*{\section}{\Large\bfseries\sffamily}
%\titleformat*{\section}{\Large\headingfont}

%\renewcommand{\headrulewidth}{0.0pt}	% no bar on top of page
%\renewcommand{\footrulewidth}{0.4pt}	% bar on bottom of page

%%% ---------------
%%% DEFINITIONS
%%% ---------------

% Define separators

% Define Title en News input
\newcommand{\JournalName}[1]{%
		\begin{center}
			%\Huge \usefont{T1}{augie}{m}{n}
            \Large \usefont{T1}{augie}{m}{n}
			#1%
		\end{center}
		\par \normalsize \normalfont}

\newcommand*{\chants}{../chants}
\newcommand*{\messe}{../messe_du_peuple_de_dieu}
\newcommand*{\pu}{../pu}
\newcommand*{\psaumes}{../psaumes}
\newcommand*{\footer}{..}

%\DefineCurrency{EUR}{name={euro}, plural={euros}, symbol={\euro}, iso={EUR}, kind=iso}

\newcommand{\NewsItem}[1]{%
\vspace{3pt}
\underline{\textbf{#1}}
	%	%\usefont{T1}{augie}{m}{n}
	%	\large \textbf{#1} %\vspace{3pt}
   %     %\Large #1 \vspace{4pt}
	%	%\par
   %     \normalsize \normalfont
		  }

\newcommand{\NewsAuthor}[1]{%
			\hfill by \textsc{#1} \vspace{4pt}
			\par \normalfont}
%\sisetup{locale=FR}
%\sisetup{group-minimum-digits=3}

\graphicspath{{../images/}}

%pas de numérotation des sections
\setsecnumdepth{none}
\setlength{\parindent}{0pt}
%%% ---------------
%%% BEGIN DOCUMENT
%%% ---------------
\begin{document}

\NewsItem{CHANT D'ENTRÉE}
	\textbf{Ouvre mes yeux, Seigneur}

1.
Ouvre mes yeux, Seigneur,
Aux merveilles de ton amour ;
Je suis l’aveugle sur le chemin,
Guéris-moi, je veux te voir.

2.
Ouvre mes mains, Seigneur,
Qui se ferment pour tout garder ;
Le pauvre a faim devant ma maison :
Apprends-moi à partager.

3.
Fais que je marche, Seigneur,
Aussi dur que soit le chemin ;
Je veux te suivre jusqu’à la croix :
Viens me prendre par la main.
%
%4.
%Fais que j’entende, Seigneur,
%Tous mes frères qui crient vers moi ;
%À leur souffrance et à leurs appels
%Que mon cœur ne soit pas sourd.
%
%5.
%Garde ma foi, Seigneur,
%Tant de voix proclament ta mort ;
%Quand vient le soir et le poids du jour,
%Ô Seigneur, reste avec moi.


\NewsItem{PRÉPARATION PÉNITENTIELLE}\\
	Kyrie \emph{messe du Peuple de Dieu}


\NewsItem{GLORIA}\\
	%Messe de Lourdes
%Gloria AL 189
GLORIA ! GLORIA ! IN EXCELSIS DEO ! (bis)

Paix sur la terre aux hommes qu’il aime.
Nous te louons, nous te bénissons, nous t’adorons.
Nous te glorifions, nous te rendons grâce pour ton immense gloire,
Seigneur Dieu, roi du ciel, Dieu le Père tout-puissant.

Seigneur, Fils unique, Jésus-Christ,
Seigneur Dieu, Agneau de dieu, le fils du Père ;
Toi qui enlèves les péchés du monde, prends pitié de nous ;
Toi qui enlèves les péchés du monde, reçois notre prière ;
Toi qui es assis à la droite du Père, prends pitié de nous.

Car Toi seul est Saint, Toi seul est Seigneur,
Toi seul est le très haut ; Jésus-Christ avec le Saint Esprit
dans la gloire de Dieu le père. Amen.



% -----
\NewsItem{1\iere{} LECTURE} Am 6, 1a.4-7
% -----

\NewsItem{PSAUME}
 Ps 145 (146), 6c.7, 8.9a, 9bc-10

\textbf{Chante, ô mon âme,
la louange du Seigneur !}

\smallskip

Le Seigneur garde à jamais sa fidélité,
il fait justice aux opprimés ;
aux affamés, il donne le pain ;
le Seigneur délie les enchaînés.

\smallskip

Le Seigneur ouvre les yeux des aveugles,
le Seigneur redresse les accablés,
le Seigneur aime les justes,
le Seigneur protège l’étranger.

\smallskip

Il soutient la veuve et l’orphelin,
il égare les pas du méchant.
D’âge en âge, le Seigneur régnera :
ton Dieu, ô Sion, pour toujours !


% -----
\NewsItem{2\ieme{} LECTURE} 1 Tm 6, 11-16

\NewsItem{ACCLAMATION}
Alleluia \emph{messe de la Bienveillance}


\NewsItem{ÉVANGILE} Lc 16, 19-31

\NewsItem{HOMÉLIE}

\NewsItem{PROFESSION DE FOI}
%JE CROIS EN TOI PÈRE, FILS ET ESPRIT.  J’AI CONFIANCE EN TOI TU ES MON AMI

    1. Père Créateur de vie, nous sommes tes enfants tu nous donnes la vie toi qui nous aime tant.

    2. Jésus né de marie, tu es le Fils de Dieu Tu nous donnes ta vie comme un cadeau précieux.

    3. Et toi Esprit de Dieu tu nous donnes ta force un souffle silencieux nous unit, nous renforce ;

    4. Je crois que je grandis en te confiant ma vie, ma famille, mes amis au nom du Père, du Fils et de l’esprit Amen. Amen. Amen.


%\newpage

\NewsItem{PRIÈRES UNIVERSELLES}
Dans ta bonté, regarde nous Dieu d’amour.


\NewsItem{OFFERTOIRE}

\NewsItem{PRIÈRES SUR LES OFFRANDES}
\textit{Nous nous levons et nous répondons : }
Que le Seigneur reçoive de vos mains ce sacrifice à la louange et à la gloire
de Son nom, pour notre bien et celui de toute l’Église.


\NewsItem{SANCTUS}
Saint le Seigneur, Saint le Seigneur,\\
Saint le Seigneur Dieu de l’univers.\\
Le ciel et la terre sont remplis de ta gloire\\
Hosanna , hosanna au plus haut des cieux !\\
Béni soit celui qui vient au nom du Seigneur\\
Hosanna, hosanna, au plus haut des cieux !


\NewsItem{ANAMNÈSE}
Nous proclamons ta mort, Seigneur ressuscité. Nous attendons ta venue dans la gloire.


\NewsItem{NOTRE PÈRE}

\NewsItem{AGNUS} \\
Agneau de Dieu Qui enlèves le péché du monde, Prends pitié de nous !  Prends pitié de nous ! (bis) \\
Agneau de Dieu Qui enlèves le péché du monde, Donne-nous la paix !  Donne-nous la paix !


\NewsItem{COMMUNION} \textbf{orgue} ou
\textbf{En accueillant l'amour}

1.  En accueillant l’amour de Jésus Christ, \\
Nous avons tout reçu des mains du Père. (bis) \\
Et nous aurons la joie de partager le pain \\
Avec les pauvres de la terre. \\
Et nous aurons la joie de partager le pain.

2.  En célébrant la mort de Jésus Christ, \\
Nous avons tout remis aux mains du Père. (bis) \\
Il nous envoie porter l’espoir du Jour qui vient \\
Parmi les pauvres de la terre. \\
Il nous envoie porter l’espoir du Jour qui vient.

3.  En devenant le corps de Jésus Christ,\\
Nous vivrons tous en fils d’un même Père. (bis)\\
Les artisans de paix témoigneront de lui\\
Auprès des pauvres de la terre.\\
Les artisans de paix témoigneront de lui.


\NewsItem{ANNONCES PAROISSIALES}


\NewsItem{CHANT D'ENVOI}

\textbf{R:
Peuple de frères, peuple du partage,
Porte l'Évangile et la paix de Dieu.}

\begin{tabular}{p{0.5\columnwidth} p{0.5\columnwidth}}
1.\newline
Dans la nuit se lèvera une lumière,\newline
L'espérance habite la terre :\newline
La terre où germera le salut de Dieu !\newline
Dans la nuit se lèvera une lumière,\newline
Notre Dieu réveille son peuple.\newline
Notre Dieu pardonne à son peuple.
&
3.\newline
La tendresse fleurira sur nos frontières,\newline
L'espérance habite la terre :\newline
La terre où germera le salut de Dieu !\newline
La tendresse fleurira sur nos frontières,\newline
Notre Dieu se donne à son peuple.
\end{tabular}



\newpage

\NewsItem{Intentions de messe}
\begin{itemize}
\item[\Cross] M. Lucien OSSWALD (Sainte Croix)
\end{itemize}

\NewsItem{Informations paroissiales}

\begin{tabular} {lcp{9cm}}
\multicolumn{3}{c}{\textbf{Saint Jean-Baptiste} } \\ \hline
Mardi    & 30 sept.  &\emph{S. Jérôme, prêtre et docteur de l'Église}\newline  Vêpres 18h15. Messe 18h30 \\ \hline
Jeudi    & 02 oct. &\emph{Ss Anges Gardiens}\newline
Exposition du Saint Sacrement à 16h00. Adoration. Salut au Saint Sacrement à 18h15. Messe à 18h30 
 \\ \hline
Vendredi & 03 oct. & Laudes 08h45. Messe 09h00 \\ \hline
Samedi   & 04 oct. & Messe anticipée 18h00 \\ \hline
Dimanche  & 05 oct. & Messe 11h00\\ \hline
\multicolumn{3}{c}{\textbf{Sainte Croix} } \\ \hline
Mercredi & 01 oct.  & \emph{Ste Thérèse de l'Enfant Jésus, vierge}\newline Messe 09h00. \\ \hline
%\newline \Cross{} \textbf{Enterrement}  14h30 Marie-Claire Noël \\ \hline
Dimanche  & 05 oct. & Messe 09h30\\ \hline
\multicolumn{3}{c}{\textbf{Résidence Landsberg (3 rue Jean Monnet)} } \\ \hline
Mercredi & 01 oct. : & Messe 10h45 \\ \hline
\end{tabular}

\begin{framed}
\begin{itemize}
\item
\textbf{Éveil à la foi} : dimanche 05 octobre à 11h00.
\item
Les \textbf{mini-rentrées} (présentation des équipes) vont avoir lieu les dimanches 28 septembre et 05 octobre.
\item
\textbf{Les familles endeuillées} sont invitées à une après-midi conviviale mardi le 30 septembre 2025 à 14h30
au presbytère St Jean-Baptiste pour un temps de partage autour d'un texte de l’Évangile et de la période difficile du deuil.
Personne contact : Brigitte FLORIAN 06.73.45.66.78
\item
Dimanche 19 octobre : \emph{envoi en mission des jeunes} 10h00 \textbf{messe unique} à Sainte Croix
\end{itemize}
\end{framed}

\NewsItem{Répétitions des chorales}
\begin{description}
\item[Chorales paroissiales] : vendredi 20h15 à Sainte Croix
\end{description}

\begin{framed}
\textbf{Presbytère St Jean-Baptiste}
%2 rue de l'école 67380 Lingolsheim 03 88 78 16 45 \\
2 rue de l'école 67380 Lingolsheim \phonenumber[country=FR]{0388781645} \\
\textbf{Permanence} Lun. au Jeu. : 09h30-12h00 et 15h-18h. Ven. 10h-12h et 15h-18h. Sam. 09h30-12h00.
\textbf{Courriel} \href{mailto:danielette67380@gmail.com}{danielette67380@gmail.com}

%\textbf{Caritas} Vestiaire ouvert le mardi de 14h à 16h

\texttt{https://stjeanbaptistelingo.fr} \hfill \faFacebook Catho Lingo \hfill \faInstagram @catho\_lingo
\end{framed}



\newpage

\JournalName{Communauté de Paroisses de Lingolsheim \\
\normalsize \textit{Notre Dame des Sables}
%\\ \large \'{E}glise Saint Jean-Baptiste
\\  \normalsize \textit{26\ieme{} dimanche du Temps Ordinaire - C}
\\ \large Dimanche 28 septembre 2025}
%\noindent\HorRule{3pt} \\[-0.75\baselineskip]
%\HorRule{1pt}
% -----

% Front article
% -----
%\vspace{0.5cm}
%	\SepRule
%\vspace{0.5cm}

%\begin{center}
\begin{minipage}[h]{1.0\linewidth}
\setlength{\parindent}{1em}
 \begin{center}
 \textbf{
 %\dots
\og 
\og Ils sont justes passés de l’autre côté\fg{}
 \fg{}
 %\dots
 }
 \end{center}

Lorsqu’un être que nous aimons, qui nous est très proche, est emporté par la mort, nous disons que nous l’avons \og perdu \fg. Et nous avons l’impression désespérante que cette perte est définitive, que la mort est vraiment une fin. Nous disons alors volontiers que nous avons \og fait une perte irréparable \fg.

En effet, c’est toujours douloureux de perdre un être cher. Nul ne peut dire le contraire. Mais chrétiens ont imbus de la parole de Dieu, la foi chrétienne nous assure que les morts ressusciteront. Nous le disons même dans notre Credo :
\og Je crois\dots{} à la résurrection de la chair, à la vie éternelle \fg{}. Cette compréhension du Credo, doit nous aider à supporter l’épreuve présente.
Nous devons le croire, même si nous ne le voyons pas. Il nous faut donc nous laisser accabler par le souvenir de la fin douloureuse de celui que nous avons perdu. La foi doit nous aider à découvrir, sous ces apparences désolantes, la croissance invisible mais réelle d’un homme nouveau, qui se dégage peu à peu de cette épreuve pour acquérir son visage de beauté et d’éternité. C’est un peu la fleur qui, alors qu’elle se dessèche et flétrit, prépare le fruit brillant et savoureux qui se noue à partir d’elle. Ainsi la mort n’est pas seulement ce qu’elle parait :
diminution et destruction. Elle est en réalité transfiguration. Et c’est bien ce que nous reprenons dans la première préface des défunts où nous disons :
\og Pour tous ceux qui croient en toi, Seigneur, la vie n’est détruite, elle est transformée ; et lorsque prend fin leur séjour sur la terre, ils ont déjà une demeure éternelle dans les cieux. \fg{}

Dans l’espérance de la mort de ceux qui nous ont devancé, si nous sommes attentifs, ne nous oblige-t-elle pas en effet à regarder les choses autrement. Ainsi pour le croyant, la vie n’est pas détruite. Elle est transformée. Notre séjour sur terre est terminé mais alors s’ouvrent pour nous les portes d’une nouvelle demeure où il n’y a plus ni larmes, ni maladie et ni souffrance ; mais la paix et la joie.

\begin{wrapfigure}{l}{1.7cm}
\vspace{-0.4cm}
	\includegraphics[scale=1.20]{../images/standing_daniel}
\end{wrapfigure}
Au cours ce mois de novembre que nous assemblées en soient déjà l’annonce et la préfiguration et que Dieu nous donne la joie de remarquer, autour de de nous, les petits signes qui montrent bien que notre prière est exaucée et que ce royaume arrive, qu’il est au milieu de nous et que bientôt nous aurons la claire vision de ce que nous espérons dans la foi.



\begin{flushright}
\textit{Père  Daniel  ETTÉ}
\end{flushright}


\end{minipage}
%\end{center}
% -----
\end{document}
