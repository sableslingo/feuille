%%% ---------------
%%% PREAMBLE
%%% ---------------
\documentclass[11pt,a4paper]{article}

% Define geometry (without using the geometry package)
\usepackage{geometry}
\geometry{landscape, twocolumn, textwidth=27.5cm, textheight=19.5cm, columnsep=12mm}

%\frenchspacing						% better looking spacing

% Call packages we'll need
\usepackage[french]{babel}			% french
\usepackage{graphicx}				% images
\usepackage{amssymb,amsmath}		% math
\usepackage{multicol}				% three-column layout
\usepackage{url}					% clickable links
\usepackage{marvosym}				% symbols
\usepackage{wrapfig}				% wrapping text around figures
\usepackage{fontspec}			% font encoding
\usepackage{xunicode}
\usepackage{ragged2e}
\usepackage{titlesec}
\usepackage{framed}
\usepackage{tocvsec2}
% Customize (header and) footer
\usepackage{fancyhdr}
\usepackage{enumitem}
%\pagestyle{fancy}
\pagestyle{empty}
\setmainfont{Carlito}

%\titlespacing\section{0pt}{0pt plus 4pt minus 2pt}{0pt plus 2pt minus 2pt}
%\titlespacing\subsection{0pt}{12pt plus 4pt minus 2pt}{0pt plus 2pt minus 2pt}
%\titlespacing\subsubsection{0pt}{12pt plus 4pt minus 2pt}{0pt plus 2pt minus 2pt}

%\newfontfamily\headingfont[]{Arial}
%\titleformat*{\section}{\Large\bfseries\sffamily}
%\titleformat*{\section}{\Large\headingfont}

%\renewcommand{\headrulewidth}{0.0pt}	% no bar on top of page
%\renewcommand{\footrulewidth}{0.4pt}	% bar on bottom of page

%%% ---------------
%%% DEFINITIONS
%%% ---------------

% Define separators
\newcommand{\HorRule}[1]{\noindent\rule{\linewidth}{#1}} % Creating a horizontal rule
\newcommand{\SepRule}{\noindent							 % Creating a separator
						\begin{center}
							\rule{250pt}{1pt}
						\end{center}
						}						

% Define Title en News input
\newcommand{\JournalName}[1]{%
		\begin{center}	
			%\Huge \usefont{T1}{augie}{m}{n}
            \Large \usefont{T1}{augie}{m}{n}
			#1%
		\end{center}	
		\par \normalsize \normalfont}
		
\newcommand{\JournalIssue}[1]{%
		\hfill \textsc{\mydate \today, No #1}
		\par \normalsize \normalfont}

\newcommand{\NewsItem}[1]{%
\vspace{3pt}
\underline{\textbf{#1}}
	%	%\usefont{T1}{augie}{m}{n} 	
	%	\large \textbf{#1} %\vspace{3pt}
   %     %\Large #1 \vspace{4pt}
	%	%\par 
   %     \normalsize \normalfont
		  }
		
\newcommand{\NewsAuthor}[1]{%
			\hfill by \textsc{#1} \vspace{4pt}
			\par \normalfont}		

%pas de numérotation des sections
\setsecnumdepth{none}
\setlength{\parindent}{0pt}
%%% ---------------
%%% BEGIN DOCUMENT
%%% ---------------
\begin{document}
% Title	
% -----




% Other news (1)
% -----
%\vspace{0.5cm}
%	\SepRule
%\vspace{0.5cm}

\begin{center}
\NewsItem{ENTRÉE} \\
SILENCE\\
PRIÈRE D'OUVERTURE
\end{center}

% -----
\NewsItem{PREMIÈRE LECTURE} Is 52, 13 – 53, 12
% -----

\begin{framed}
\NewsItem{PSAUME} 30 (31), 2ab.6, 12, 13-14ad, 15-16, 17.25
%\begin{itemize}
%\item[R/]
%Ô Père, en tes mains
%je remets mon esprit.
%\item[]
%En toi, Seigneur, j’ai mon refuge ;
%garde-moi d’être humilié pour toujours.
%En tes mains je remets mon esprit ;
%tu me rachètes, Seigneur, Dieu de vérité.
%\item[]
%Je suis la risée de mes adversaires
%et même de mes voisins ;
%je fais peur à mes amis,
%s’ils me voient dans la rue, ils me fuient.
%\item[]
%On m’ignore comme un mort oublié,
%comme une chose qu’on jette.
%J’entends les calomnies de la foule :
%ils s’accordent pour m’ôter la vie.
%\item[]
%Moi, je suis sûr de toi, Seigneur,
%je dis : « Tu es mon Dieu ! »
%Mes jours sont dans ta main : délivre-moi
%des mains hostiles qui s’acharnent.
%\item[]
%Sur ton serviteur, que s’illumine ta face ;
%sauve-moi par ton amour.
%Soyez forts, prenez courage,
%vous tous qui espérez le Seigneur !
%\end{itemize}

R/ Ô Père, en tes mains
je remets mon esprit.

\smallskip
En toi, Seigneur, j’ai mon refuge ;
garde-moi d’être humilié pour toujours.
En tes mains je remets mon esprit ;
tu me rachètes, Seigneur, Dieu de vérité.

\smallskip
Je suis la risée de mes adversaires
et même de mes voisins ;
je fais peur à mes amis,
s’ils me voient dans la rue, ils me fuient.

\smallskip
On m’ignore comme un mort oublié,
comme une chose qu’on jette.
J’entends les calomnies de la foule :
ils s’accordent pour m’ôter la vie.

\smallskip
Moi, je suis sûr de toi, Seigneur,
je dis : « Tu es mon Dieu ! »
Mes jours sont dans ta main : délivre-moi
des mains hostiles qui s’acharnent.

\smallskip
Sur ton serviteur, que s’illumine ta face ;
sauve-moi par ton amour.
Soyez forts, prenez courage,
vous tous qui espérez le Seigneur !

\end{framed}

\NewsItem{DEUXIÈME LECTURE} He 4, 14-16 ; 5, 7-9

\NewsItem{ACCLAMATION}
	%\begin{itemize}
%\item[]
%Gloire au Christ parole éternelle du Dieu vivant !
%\item[]
%Gloire à toi Seigneur !
%\end{itemize}
Gloire au Christ parole éternelle du Dieu vivant !
Gloire à toi Seigneur !


% -----
\begin{framed}
\NewsItem{ÉVANGILE} Passion de notre Seigneur Jésus Christ (Jn 18, 1 – 19, 42)

\texttt{Chant durant la passion}  R. : Mon peuple, que t'ai-je fait ? En quoi
t'ai-je offensé ? Réponds-moi ! Ô Dieu Saint, ô Dieu Saint, fort.
Ô Dieu Saint, ô Dieu Saint, fort, immortel, prends pitié de nous.
\end{framed}
\NewsItem{HOMÉLIE}

\NewsItem{LES GRANDES PRIÈRES}

Christ mort sur la croix, exauce notre prière.

\NewsItem{VÉNÉRATION DE LA CROIX} 
\begin{framed}
O Crux ave, spes unica hoc passionis tempore hoc passionis tempore
Auge piis justitiam Reisque dona veniam. Auge piis justitiam
Reisque dona veniam. O Crux ave, spes unica hoc passionis tempore hoc
passionis tempore
\end{framed}

\NewsItem{PRÉSENTATION DE LA CROIX} 

\NewsItem{VÉNÉRATION DES FIDÈLES}
\begin{framed}
\begin{itemize}
\item[1.] En ce jour est crucifié Le Créateur du monde,
Il est couronné, Lui, le Roi des cieux. Il est attaché au bois,
L’Époux de l’Église. Nous adorons tes souffrances, Ô Christ notre Dieu.
\item[R.] \textbf{Ô Seigneur, prends pitié de nous, Par ta croix sauve-nous !}
\item[2.] Devant toi, Seigneur Jésus, Tout tremble et se prosterne,
Et que toute langue chante Que tu es Seigneur. Tu acceptes nos souffrances
Pour nous racheter. Tu nous laves par ton sang, Efface nos péchés.
\item[3.] Toi qui, élevé en croix, Détruis tous les enfers. Tu effaces la sentence portée contre tous. Obtiens-nous la pénitence De ton bon larron.
Ô Seigneur, dans ton royaume, De nous souviens toi !
\end{itemize}
\end{framed}

\NewsItem{OFFERTOIRE} Voici l'Homme
\begin{itemize}
\item[R/] : Jésus Christ roi blessé, Dieu couronné de nos épines.
Ô Seigneur prends pitié, que ton pardon nous illumine.
\item[1.] L’homme, voici l’homme, jamais homme n’a parlé comme cet homme.
Roi de silence, roi qui se tait devant l’offense, roi de patience et de beauté.
\item[2.] L’homme, voici l’homme, jamais homme ne fut vrai comme cet homme.
Roi de lumière, roi humilié dans la poussière, roi de prière et de clarté.
\item[3.] L’homme, voici l’homme, jamais homme n’a aimé comme cet homme.
Roi de largesse, roi qui console nos détresses, roi de tendresse en nos duretés.
\end{itemize}

\NewsItem{NOTRE PÈRE}


\begin{framed}
\NewsItem{COMMUNION} Berger du silence

Silence de Dieu, au jardin d’agonie, silence de Dieu qui rend la nuit plus noire.
Silence de Dieu, quand la coupe est à boire, Tu es l’enfantement d’une autre vie où la mort est changée, un matin en victoire.
Dieu, berger du silence...

\smallskip
Silence de Dieu, quand l’Arbre meurt en croix, silence de Dieu, tu deviens
cette sève. Silence de Dieu, qui vit et qui relève, tu donnes le fruit mûr du
Golgotha, qui germe en son tombeau, se redresse et se lève.
Dieu, berger du silence...

\smallskip
Silence de dieu, qui alourdit nos croix, silence de Dieu, au temps de nos
souffrances. Silence de Dieu, qui ressemble à l’absence, tu es saison d’hiver et
de vents froids, où germe en notre sol, lentement, ta Semence.
Dieu, berger du silence...


\end{framed}

\NewsItem{PRIÈRE DE CONCLUSION}

\NewsItem{PRIÈRE DE BÉNÉDICTION}

\NewsItem{CHANT D'ENVOI}
\textbf{Victoire, tu régneras, ô Croix, tu nous sauveras.}

\begin{tabular}{p{0.5\columnwidth} p{0.5\columnwidth}}
1.
Rayonne sur le monde qui cherche la Vérité.
Ô Croix, source féconde d'amour et de liberté.
&
2.
Redonne la vaillance au pauvre et au malheureux.
C'est toi notre espérance qui nous mènera vers Dieu.
\end{tabular}
3.
Rassemble tous nos frères à l'ombre de tes grands bras.\\
Par toi Dieu notre Père, au ciel, nous accueillera. 


\NewsItem{SORTIE EN SILENCE}


\NewsItem{Informations paroissiales}

\textit{Les horaires des messes et des célébrations sont affichés à l’extérieur des églises, sur Internet et les réseaux sociaux Facebook et Instagram.}



\begin{framed}
\begin{description}
\item[Célébrations de Pâques]
~\\
Samedi 19 avril              : Vigile Pascale à 20h30 à l’église Sainte-Croix\\
Dimanche 20 avril         : Messe de Pâques à 10h à l’église Saint Jean-Baptiste
\item[Rencontre des confirmands avec Mgr Pascal DELANNOY]
Mardi 22 avril de 18h30 à 20h à l’Archevêché 16 Rue Brûlée, 67000 Strasbourg
\item[Voyage en Côte d’Ivoire du 8 au 17 février 2026]
~\\
Mercredi 23 avril : réunion d’information concernant le voyage en Côte d’Ivoire
organisé avec le Diocèse de Grand Bassam à un tarif attractif.
Inscription par mail avant \underline{le 30 avril 2025} à : \texttt{paule.troestler@gmail.com} et/ou   \texttt{nathalie\_dick@yahoo.fr}
\end{description}
\end{framed}


\textbf{Répétitions des chorales}\\
\textbf{Chorales paroissiales} : vendredi 20h15 à Sainte Croix\\
\textbf{Groupe \og Dans l’Amour de Dieu \fg} : samedi 16h30 à Saint Jean-Baptiste

\begin{multicols}{2}
\textbf{Presbytère Saint Jean-Baptiste}\\
2 rue de l'école 67380 Lingolsheim 03 88 78 16 45\\
\textbf{Permanence :} Vendredi de 10h à 12h et de 17h30 à 19h\\
\textbf{Courriel :} \texttt{danielette67380@gmail.com}\\
\textbf{site internet :} \texttt{stjeanbaptistelingo.fr}\\
\textbf{Instagram :} \texttt{@catho\_lingo}\\
\textbf{Facebook :} \texttt{Catho lingo}\\
\textbf{Permancance Caritas}\\ Vestiaire ouvert le mardi de 14h à 16h\\
\end{multicols}


%         \begin{tabular}{l l l}
%         \multicolumn{3}{c}{\textbf{St Jean-Baptiste}} \\
%  Mardi & 11 fév. & Vêpres 18h15 - 18h30. Pas de messe \\
%Jeudi & 13 fév. & Salut au Saint Sacrement 18h15. Messe 18h30. \\
%    Vendredi & 14 fév. & Laudes 08h45 - 09h00. Pas de messe \\
%        Samedi  & 15 fév. & Messe anticipée 18h00 \\
%    Dimanche & 16 fév. & Pas de messe \\      
%      
%         \multicolumn{3}{c}{\textbf{Ste Croix}} \\
%         Mercredi & 12 fév. & Pas de messe \\ 
%         Dimanche & 16 fév.& Messe 10h30 \\
%    
%        \end{tabular}
  

\newpage

\JournalName{Communauté de Paroisses de Lingolsheim \\
\normalsize \textit{Notre Dame des Sables}
\\ \large \'{E}glise Saint Jean-Baptiste
\\  \normalsize \textit{Vendredi 18 avril 2025}
\\ \large Vendredi Saint}
%\noindent\HorRule{3pt} \\[-0.75\baselineskip]
%\HorRule{1pt}
% -----

% Front article
% -----
%\vspace{0.5cm}
%	\SepRule
%\vspace{0.5cm}

%\begin{center}
\begin{minipage}[h]{1.0\linewidth}
\og \textit{la Croix, signe d’espérance} \fg

Bien conscient du drame qui se trame contre lui et le bouleverse au plus profond de son être à l’approche de son ‘‘heure’’, Jésus décide pourtant d’aller jusqu’au bout du don de soi. Si au plan des relations humaines, en ce qui fait la valeur de l’homme c’est sa réussite sociale, aux yeux de Dieu l’homme ne vaut que par sa capacité de don. 

C’est bien ce qu’a été toute la vie de Jésus. Dans le don de sa vie, il a obéi à son Père jusqu’à la croix. C’est en cela que la croix reste pour nous les chrétiens un mystère. La victoire de la croix, la victoire de l'amour sur la haine et la violence, de la vérité sur le mensonge, de la vie sur la mort, demeure encore pour le mystère de notre rédemption. 

Dans le monde règnent encore aujourd'hui la haine, le mensonge et la violence. La vie nouvelle ne nous est donnée que sous la forme de la croix. "L'histoire de l'espérance, dans laquelle Jésus se manifeste comme le Fils de Dieu vivant pour toujours, n'est pas une série continue de succès, ni un récit de victoires à l'échelle humaine". C'est au bout du chemin de la croix que nous est promise la victoire de la croix. Car c'est précisément la kénose, c’est-à-dire l'abaissement de Dieu, jusque dans l'abîme de la souffrance et de la mort où nous sommes plongés, qui nous a de nouveau unis à Dieu dans cette situation concrète qui est la nôtre. Ainsi la croix est-elle signe de l'espérance que nous serons un jour totalement libérés et que Dieu définitivement règnera sur le mal.

Dès lors,
prendre notre croix quotidienne signifie nous engager dans la logique paradoxale du renoncement à soi pour le règne grandisse en nous.
\begin{wrapfigure}{l}{1.3cm}
\vspace{-0.5cm}
	\includegraphics{daniel_ette.png}
\end{wrapfigure}
Si nos sociétés sont si facilement prisonnières de la violence et des hostilités fratricides c’est parce qu’elles sont régies par la mentalité de l’égoïsme absolutisé : tout ce qui n’est pas fait pour soi semble être une perte, un échec. Seul l’intérêt, avoué ou caché, semble faire mouvoir le monde. Et pourtant c’est le don de 
soi qui libère, c’est l’amour qui sauve le monde, c’est de la Croix que jaillit la vie nouvelle.                                                                                                                                                                

\begin{flushright}
Bonne montée vers Pâques !\\
\textit{Père  Daniel  ETTÉ}
\end{flushright}

\end{minipage}
%\end{center}
% -----
\end{document} 
