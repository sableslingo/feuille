 \begin{center}
 \textbf{
 %\dots
 AVENT : DIEU VIENT !
 %\dots
 }
 \end{center}

Une nouvelle Année Liturgique commence : \emph{L’Année A} et celle-ci est marquée par le temps de l’Avent, un temps fort, qui dure quatre semaines, où l’Église entière est dans l’attente du Messie.
Vivre chaque année liturgique c’est actualiser concrètement dans cette année nouvelle le salut qu’il accompli. Le mot \og AVENT \fg{} vient du latin \og adventus \fg{}, qui signifie \og avènement, arrivée, venue \fg{},
ce qui advient sans cesse dans l’histoire humaine et celle de chacun, et ce qui est encore en attente d’une plénière. Le salut accompli par le Christ se fait avènement permanent par la puissance de l’Esprit de sa résurrection.
L’Avent nous aide à garder aussi l’Esperance chrétienne. Il est moins long et moins rigoureux que le Carême.
Durant, l’Avent, nous sommes plongés dans un climat de veille, d’attente qui débouche sur l’espérance. C’est une attente joyeuse d’accueillir l’Enfant-Dieu. Puisque nous attendons aussi un retour du Christ dans la gloire, cette attente se veut joyeuse et tournée vers l’avenir, un avenir plein d’espérance.
Jésus est venu, mais il vient maintenant, il est là constamment, notamment dans les sacrements, par exemple dans l’eucharistie pour nous faire participer à son projet d’amour de paix et de communion avec lui. Il vient maintenant pour nous donner un avant-goût de la gloire que nous partagerons totalement avec lui quand il reviendra à la parousie.
L’Avent l’Église propose plusieurs figures bibliques importantes qui nous aident à progresser dans notre préparation à Noël.

\textbf{Le premier dimanche de l’Avent} nous rencontrons la figure d’Isaïe. Il présente la promesse de Dieu qui assure son Peuple qu’il va venir lui-même pour le sauver et pour instaurer un royaume de Paix et de Justice.
\og Venez, famille de Jacob, marchons à la Lumière du Seigneur \fg{} (Isaïe 2,5).
%\begin{wrapfigure}{l}{1.7cm}
%\vspace{-0.4cm}
%	\includegraphics[scale=1.20]{../images/standing_daniel}
%\end{wrapfigure}

\textbf{Le deuxième dimanche de l’Avent}, nous contemplons Jean-Baptiste, le cousin de Jésus. Il prêche un baptême de conversion. Il nous montre que le Seigneur vient et que nous devons nous préparer à le recevoir.
\og Préparez le chemin de Seigneur, aplanissez sa route \fg{} (Mt 3,3). Il faut donc vouloir changer de vie pour être prêt à recevoir \og Celui qui vient \fg{}.

\textbf{Le troisième dimanche de l’Avent}, c’est encore Jean-Baptiste que nous contemplons, mais cette fois-ci, il nous permet de voir que le Seigneur a commencé son œuvre, il est déjà là pour changer et sauver le monde.

\textbf{Le quatrième dimanche de l’Avent}, le plus proche de Noël, nous sommes appelés à contempler la Vierge Marie, celle qui a dit \og oui \fg{} ! C’est la certitude que Dieu est bien \og l’Emmanuel \fg{} ce qui signifie \og Dieu avec nous \fg{}.
Car grâce à Marie, Jésus est vraiment Vrai Homme et Vrai Dieu. Il s’est uni à nous de manière définitive. C’est la source de notre espoir.



\begin{flushright}
				Belle Année Liturgique et Bon Temps d’Avent !!!
\textit{Père  Daniel  ETTÉ}
\end{flushright}

