%%% ---------------
%%% PREAMBLE
%%% ---------------
\documentclass[french,11pt]{article}

% Define geometry (without using the geometry package)
\usepackage[a4paper]{geometry}
\geometry{landscape, twocolumn, textwidth=28.2cm, textheight=19.5cm, columnsep=16mm}

%\frenchspacing						% better looking spacing

% Call packages we'll need
\usepackage{graphicx}				% images
\usepackage{multicol}
\usepackage{multirow}
\usepackage{url}					% clickable links
\usepackage{marvosym}				% symbols
\usepackage{wrapfig}				% wrapping text around figures
\usepackage{fontspec}			% font encoding
\usepackage{xunicode}
\usepackage{phonenumbers}
\usepackage[hidelinks]{hyperref}
\usepackage{ragged2e}
\usepackage{pst-barcode}
\usepackage{titlesec}
\usepackage{framed}
%\usepackage[default]{raleway}
\usepackage{tocvsec2}
% Customize (header and) footer
\usepackage{fancyhdr}
\usepackage{enumitem}
\usepackage{tabularx}
\usepackage{fontawesome}
\usepackage{lipsum}
\usepackage{babel}
%\usepackage{currency}
%\pagestyle{fancy}
\pagestyle{empty}
\setmainfont{Carlito}
\usepackage{caption}
\captionsetup[table]{labelformat=empty}
\setlist{nosep,noitemsep}

%\newfontfamily\headingfont[]{Arial}
%\titleformat*{\section}{\Large\bfseries\sffamily}
%\titleformat*{\section}{\Large\headingfont}

%\renewcommand{\headrulewidth}{0.0pt}	% no bar on top of page
%\renewcommand{\footrulewidth}{0.4pt}	% bar on bottom of page

%%% ---------------
%%% DEFINITIONS
%%% ---------------

% Define separators

% Define Title en News input
\newcommand{\JournalName}[1]{%
		\begin{center}
			%\Huge \usefont{T1}{augie}{m}{n}
            \Large \usefont{T1}{augie}{m}{n}
			#1%
		\end{center}
		\par \normalsize \normalfont}

\newcommand*{\chants}{../chants}
%\newcommand*{\messe}{../messe_grace}
\newcommand*{\messe}{../messe_soleil_des_nations}
\newcommand*{\pu}{../pu}
\newcommand*{\psaumes}{../psaumes}
\newcommand*{\footer}{..}

%\DefineCurrency{EUR}{name={euro}, plural={euros}, symbol={\euro}, iso={EUR}, kind=iso}

\newcommand{\presbytere}{\textbf{Presbytère}}
\newcommand{\sjb}{\textbf{St Jean-Baptiste}}
\newcommand{\sx}{\textbf{Ste Croix}}
\newcommand{\landsberg}{\textbf{Landsberg}}

\newcommand{\NewsItem}[1]{%
\vspace{3pt}
\underline{\textbf{#1}}
	%	%\usefont{T1}{augie}{m}{n}
	%	\large \textbf{#1} %\vspace{3pt}
   %     %\Large #1 \vspace{4pt}
	%	%\par
   %     \normalsize \normalfont
		  }

\newcommand{\NewsAuthor}[1]{%
			\hfill by \textsc{#1} \vspace{4pt}
			\par \normalfont}
%\sisetup{locale=FR}
%\sisetup{group-minimum-digits=3}

\graphicspath{{../images/}}

%pas de numérotation des sections
\setsecnumdepth{none}
\setlength{\parindent}{0pt}
%%% ---------------
%%% BEGIN DOCUMENT
%%% ---------------
\begin{document}

\begin{center}
LUMIÈRE DE LA PAIX DE BETHLÉEM - 2025
\end{center}

\NewsItem{CHANT D'ENTRÉE}\\
	VEILLEURS, BÉNISSEZ DIEU DANS LA NUIT, \emph{(geste du guetteur, main sur le front)} \\
IL NOUS DONNE SA PAIX \emph{(on se donne la main)}\\
VEILLEURS, BÉNISSEZ DIEU, ÉLEVEZ LES MAINS, \emph{(on élève les deux mains vers le ciel)} \\
DANS LA NUIT, BÉNISSEZ SANS FIN.


1.
Dans le silence,
Faites monter en vos cœurs,
La joie, la louange.

2.
Gardez vos lampes
Allumées pour le retour de Dieu,
Notre maître.

3.
Dans la confiance,
Présentez au Seigneur votre encens,
Vos prières. 


\NewsItem{TEMPS DE L'OBSCURITÉ}

\NewsItem{CHANT} Dans nos obscurités (répons de Taizé)

\smallskip

\emph{Arrivée de la lumière par les scouts}

\smallskip

\NewsItem{CHANT} 
\textbf{La paix, elle aura ton visage,\\
La paix, elle aura tous les âges.\\
La paix sera toi, sera moi, sera nous,\\
Et la paix sera chacun de nous.}


% -----
\NewsItem{1\iere{} LECTURE} Is 35, 1-6a.10
% -----

\NewsItem{PSAUME}
\textbf{
Peuples qui marchez dans la longue nuit,\\
Le jour va bientôt se lever.\\
Peuples qui cherchez le chemin de vie\\
Dieu lui-même vient vous sauver,\\
Peuples qui cherchez le chemin de vie\\
Dieu lui-même vient vous sauver.
}

1.
Il est temps de lever les yeux,\\
Vers le monde qui vient.\\
Il est temps de jeter la fleur,\\
Qui se fane en vos mains.
%2
%Il est temps de tuer la peur,
%Qui vous garde en ses liens.
%Il est temps de porter la Croix,
%Jusqu'au bout du chemin.
%3
%Il est temps de bâtir la paix,
%Dans ce monde qui meurt.
%Il est temps de laisser l´amour
%Libérer votre cœur.
%4
%Il est temps de laisser les morts
%S´occuper de leurs morts.
%Il est temps de laisser le feu
%Ranimer votre cœur. 


% -----
\NewsItem{2\ieme{} LECTURE} Jc 5, 7-10

\NewsItem{ACCLAMATION}
%Alleluia \emph{messe de la Bienveillance}


\NewsItem{ÉVANGILE} Mt 11, 2-11

\NewsItem{PRÉDICATION}

\newpage

\NewsItem{DÉCLARATION DE FOI COMMUNE}

Je crois en Dieu, le Père qui appelle à la vie, qui désire et inspire chaque vie.
Je crois en Dieu, le Père, qui croit en moi, qui habite en moi, qui a besoin de tous les hommes pour répandre sa Paix.
Je crois en Dieu, le Père, confident, artisan de communion, qui nous laisse libre de le choisir et de l’aimer.
Je crois au Père tout-puissant d’Amour.

\medskip

Je crois en Jésus Christ, son fils, notre frère, qui a pris corps d’homme, pour nous parler droit au cœur, nous permettre de vivre pleinement et dans la confiance de la Foi.
Je crois en Jésus Christ, son fils, lumière et joie du Père, venu ouvrir les yeux du monde par sa Parole d’Amour vivifiante.
Je crois en Jésus-Christ, son fils, ressuscité au cœur de nos réconciliations et de nos rassemblements.

\medskip

Je crois en l’Esprit Saint, vraie présence du Père et du Fils en nos vies, regard nouveau, brise revigorante qui fait de l’Amour un monde sans frontière.
Je crois en l’Esprit Saint, force apaisante et consolatrice dans nos tempêtes, guide au cœur de notre quotidien.
Je crois en Toi, Esprit Saint, lien entre Dieu et l’homme, tu chuchotes à notre oreille, et nos promesses prennent un sens.


%\newpage

\NewsItem{PRIÈRE D'INTERCESSION}\\
Seigneur fais de nous des ouvriers de paix,\\
Seigneur fais de nous des bâtisseurs d’amour.

\NewsItem{NOTRE PÈRE}

\NewsItem{OFFRANDE}


\NewsItem{ANNONCES PAROISSIALES}


\NewsItem{CHANT FINAL}\\
\textbf{
Une flamme en moi réchauffe mon cœur. \\
Cette flamme en moi brûle mes malheurs. \\
Je sens qu'elle est là : sa douce lueur \\
Brille en moi, brille en moi, brille en moi.
}

1. C'est une flamme d'amour Qui m'éclaire dans la nuit \\
Et cette lumière d'amour Vient illuminer ma vie. \\
En attendant le matin, Je vais dire à mes voisins \\
Que cette lumière, enfin, Vient briller en moi.

2. Le lundi j'ai son amour, Le mardi je prends sa paix ; \\
Mercredi est un beau jour ; Jeudi je veux le chanter ; \\
Vendredi, si j'ai douté, Samedi, il vient m'aider ; \\
Le dimanche cette joie Vient briller en moi.



\newpage

\begin{table}[!h]
\centering
%\caption{Informations paroissiales}
\begin{tabularx}{\columnwidth}{|p{0.1\columnwidth}p{0.1\columnwidth}|p{0.19\columnwidth}p{0.07\columnwidth}X|}
\hline
%Mardi &  18 nov.  & \multicolumn{2}{c|}{ \emph{S. Martin de Tours, évêque} } \\
Mardi & 16 déc. & \sjb & 18h15 & \textbf{pas de vêpres} \\
& &      & 18h30 & \textbf{pas de messe} \\
%& &      & 18h30 & Messe \\
\hline
%Mercredi &  10 déc.  & \multicolumn{2}{c|}{ \emph{S. François Xavier, prêtre} } \\
%& & \sx		& Messe 09h00 \\
Mercredi & 17 déc. & \sx & 09h00 & Messe \\
\hline
%Jeudi &  11 déc. &		\presbytere & Confessions 16h00 - 17h30 \\
%Jeudi &  18 déc. &	\sjb &Salut au Saint Sacrement 18h15 \\
%& &		&Messe 18h30 \\
Jeudi &  18 déc. & \sx & 09h00 & Célébration pénitentiaire \\
&  &  & 09h30 & Confession \\
&  & 	\sjb&18h15 &Salut au Saint Sacrement  \\
& & &18h30	&Messe \\
\hline
%Vendredi &  28 nov.  & \multicolumn{2}{c|}{ \emph{Présentation de la Bienheureuse Vierge Marie} } \\
Vendredi &  19 déc.  & 	\sjb &08h45 & Laudes \\
& &&09h00		& Messe \\
\hline
Samedi & 20 déc.  & 	\presbytere & 15h00 & \emph{pas de confession} \\
& & 	\sjb & 18h00 & Messe anticipée \\
\hline
Dimanche &  21 déc.  & 	\sx & 09h30 & Messe\\
%Dimanche &  21 déc.  & &\multicolumn{2}{c|}{ \emph{4\ieme{} Dimanche de l'Avent} } \\
%Dimanche &  16 nov.  &		\sx & Messe 09h30 \\
\multicolumn{2}{|l|}{ \emph{4\ieme{} Dimanche de l'Avent} }  &		\sjb & 11h00 & Messe \\
\hline
\hline
Mercredi & 24 déc.  & 	\sjb & 16h00 & Messe de la nuit anticipée \\
& & 	& \multicolumn{2}{r|}{ animée par Véronique Reinbold}\\
& & 	\sx & 16h00 & Messe de la nuit anticipée avec crèche vivante \\
\hline
Jeudi & 25 déc.  & 	\sjb & 00h00 & Messe de la nuit \\
\multicolumn{2}{|l|}{ \emph{Nativité du Seigneur} }  &		\sx & 10h00 & Messe du jour \\
\hline
Vendredi & 26 déc.  & \multicolumn{3}{|c|}{ \emph{Saint Étienne, premier martyr} }	\\
& & 	\sjb & 08h45 & Laudes \\
& & 	& 09h00 & Messe \\
\hline
\end{tabularx}
\end{table}

%\begin{framed}
\begin{itemize}
\item
\textbf{Fête de Noël des enfants} sam. 20 déc. de 14h30 à 17 à Ste Croix. Activités et goûter. Tous les enfants sont les bienvenus.
\item
\textbf{Concert} avec la chorale CHORENSTRA et la chorale des hommes de Molsheim le dim. 21 déc. à 17h00 à \sjb.
%\item
%\textbf{Confessions} au presbytère : les 1\iers{} jeudis du mois de 16h à 17h30. Les 3\iemes{} samedis du mois de 15h à 16h30.
\end{itemize}
%\end{framed}

\NewsItem{Répétitions des chorales}
\begin{description}
\item[Chorales paroissiales] : vendredi 20h15 à Sainte Croix
\end{description}

\begin{framed}
\textbf{Presbytère St Jean-Baptiste}
%2 rue de l'école 67380 Lingolsheim 03 88 78 16 45 \\
2 rue de l'école 67380 Lingolsheim \phonenumber[country=FR]{0388781645} \\
\textbf{Permanence} Lun. au Jeu. : 09h30-12h00 et 15h-18h. Ven. 10h-12h et 15h-18h. Sam. 09h30-12h00.
\textbf{Courriel} \href{mailto:danielette67380@gmail.com}{danielette67380@gmail.com}

%\textbf{Caritas} Vestiaire ouvert le mardi de 14h à 16h

\texttt{https://stjeanbaptistelingo.fr} \hfill \faFacebook Catho Lingo \hfill \faInstagram @catho\_lingo
\end{framed}



\newpage

\JournalName{Communauté de Paroisses de Lingolsheim 
%\\ \normalsize \textit{Notre Dame des Sables}
%\\ \large \'{E}glise Saint Jean-Baptiste
\\  \normalsize \textit{3\ieme{} Dimanche de l'Avent, de Gaudete - A}
- \large Dim. 14 déc. 2025}
%\\ \large Dimanche 30 novembre 2025}
%\noindent\HorRule{3pt} \\[-0.75\baselineskip]
%\HorRule{1pt}
% -----

% Front article
% -----
%\vspace{0.5cm}
%	\SepRule
%\vspace{0.5cm}

%\begin{center}
\begin{minipage}[h]{1.0\linewidth}
\setlength{\parindent}{1em}
 \begin{center}
 \textbf{
 %\dots
\og 
\og Ils sont justes passés de l’autre côté\fg{}
 \fg{}
 %\dots
 }
 \end{center}

Lorsqu’un être que nous aimons, qui nous est très proche, est emporté par la mort, nous disons que nous l’avons \og perdu \fg. Et nous avons l’impression désespérante que cette perte est définitive, que la mort est vraiment une fin. Nous disons alors volontiers que nous avons \og fait une perte irréparable \fg.

En effet, c’est toujours douloureux de perdre un être cher. Nul ne peut dire le contraire. Mais chrétiens ont imbus de la parole de Dieu, la foi chrétienne nous assure que les morts ressusciteront. Nous le disons même dans notre Credo :
\og Je crois\dots{} à la résurrection de la chair, à la vie éternelle \fg{}. Cette compréhension du Credo, doit nous aider à supporter l’épreuve présente.
Nous devons le croire, même si nous ne le voyons pas. Il nous faut donc nous laisser accabler par le souvenir de la fin douloureuse de celui que nous avons perdu. La foi doit nous aider à découvrir, sous ces apparences désolantes, la croissance invisible mais réelle d’un homme nouveau, qui se dégage peu à peu de cette épreuve pour acquérir son visage de beauté et d’éternité. C’est un peu la fleur qui, alors qu’elle se dessèche et flétrit, prépare le fruit brillant et savoureux qui se noue à partir d’elle. Ainsi la mort n’est pas seulement ce qu’elle parait :
diminution et destruction. Elle est en réalité transfiguration. Et c’est bien ce que nous reprenons dans la première préface des défunts où nous disons :
\og Pour tous ceux qui croient en toi, Seigneur, la vie n’est détruite, elle est transformée ; et lorsque prend fin leur séjour sur la terre, ils ont déjà une demeure éternelle dans les cieux. \fg{}

Dans l’espérance de la mort de ceux qui nous ont devancé, si nous sommes attentifs, ne nous oblige-t-elle pas en effet à regarder les choses autrement. Ainsi pour le croyant, la vie n’est pas détruite. Elle est transformée. Notre séjour sur terre est terminé mais alors s’ouvrent pour nous les portes d’une nouvelle demeure où il n’y a plus ni larmes, ni maladie et ni souffrance ; mais la paix et la joie.

\begin{wrapfigure}{l}{1.7cm}
\vspace{-0.4cm}
	\includegraphics[scale=1.20]{../images/standing_daniel}
\end{wrapfigure}
Au cours ce mois de novembre que nous assemblées en soient déjà l’annonce et la préfiguration et que Dieu nous donne la joie de remarquer, autour de de nous, les petits signes qui montrent bien que notre prière est exaucée et que ce royaume arrive, qu’il est au milieu de nous et que bientôt nous aurons la claire vision de ce que nous espérons dans la foi.



\begin{flushright}
\textit{Père  Daniel  ETTÉ}
\end{flushright}


\end{minipage}
%\end{center}
% -----
\end{document}
